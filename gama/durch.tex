\section{Measurement}

\subsection{Calibration of the Instruments and the NaJ-Scintillator}
Our tutor instructed us how to use the instruments of the lab. During the experiment we worked with a Scintillation-counter, an Amplifier, an Oscilloscope and a Computer with a Software called 'Cassy Lab'. At the beginning we had to calibrate the instruments with the radioactive nuclides of $^{60}$Co, $^{137}$Cs and $^{22}$Na. After adjusting the settings of the Oscilloscope and the Amplifier, we took the $^{60}$Co sample in front of the NaJ-Detector, and plotted the spectrum of the sample with the software. We repeated this step with the other two samples. Each of the samples were detected for three minutes so the software generates a plot where we could see the different peaks of the samples (for a closer look on our results, see the following images).

\myImage[10cm]{img/cs137}{The maximum of the fit is at channel (6114.98$\pm$0.68) containing the energy 661.66keV}

The following picture is an illustration of a closer look at the photopeak of $^{60}$Co with the Gauschen fit.

\myImage[10cm]{img/na22_zoom}{The maximum of the fit is at channel (5642.27$\pm$5.66) containing the energy 1274.55keV}

After fitting the photopeaks of $^{60}$Co (has 2 ones), $^{137}$Cs and $^{22}$Na we plot the result in the following picture to illustrate the calibration of the NaJ-Scintillator-Detector.

Using these values we get an energy resolution

\subsection{Absorptions coefficient of different materials}
To determine the absorption coefficient of different materials, we took a different number of plates of different materials. These materials had been Aluminium, Lead and Copper. The plates had have a thickness between 2 and 3 mm so in the first place we took only no plate to calibrate and proceeded until we had seven of them. We started with the Copper plates, the we measured the absorption of Aluminium and finished with lead.

\subsection{Calibration of the Ge-Detector}
To calibrate the Ge-detector we took again $^{60}$Co, $^{137}$Cs and $^{22}$Na

\subsection{1 hour measurement of $^{137}$Cs}
To get a high resolution of a spectrum of $^{137}$Cs, we took the probe and measured the radiation for one hour. Our result is shown in the following graphic.


\subsection{Measurement of X-Ray conversion-lines}
For this experiment we took 