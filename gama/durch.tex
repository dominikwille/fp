\section{Measurement}

\subsection{Task 1 and 2 - Calibration of the Instruments}
Our tutor instructed us how to use the Instruments of the lab. During the Experiment we worked with a Scintillation counter, an Amplifier, an Oscilloscope and a Computer with a Software called 'Cassy Lab'. At the beginning we had to calibrate the Instruments with the radioactive Nuclides of $^{60}$Co, $^{137}$Cs and $^{22}$Na. After adjusting the settings of the Oscilloscope and the Amplifier, we took the $^{60}$Co sample in front of the NaJ-Detector and plotted the spectrum of the sample with the software. We repeated this step with the other two samples. Each of the samples were detected for three minutes so the software generates a plot where we could see the different Peaks of the samples (for a closer look on our results, see the following images).

\subsection{Task 3 - Absorptions coefficient of different materials}
To determine the absorption coefficient of different materials, we took a different number of plates of different materials. These materials had been Aluminium, Lead and Copper. The plates had have a thickness between 2 and 3 mm so in the first place we took only one plate and proceeded until we had seven of them. 

