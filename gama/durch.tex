\section{Measurement}

\subsection{Calibration of the Instruments and the NaJ-Scintillator}
Our tutor instructed us how to use the instruments of the lab. During the experiment we worked with a Scintillation-counter, an Amplifier, an Oscilloscope and a Computer with a Software called 'Cassy Lab'. At the beginning we had to calibrate the instruments with the radioactive nuclides of $^{60}$Co, $^{137}$Cs and $^{22}$Na. After adjusting the settings of the Oscilloscope and the Amplifier, we took the $^{60}$Co sample in front of the NaJ-Detector, and plotted the spectrum of the sample with the software. We repeated this step with the other two samples. Each of the samples were detected for three minutes so the software generates a plot where we could see the different peaks of the samples (for a closer look on our results, see the following images).

\myImage[10cm]{img/cs137}{The maximum of the fit is at channel (6114.98$\pm$0.68) containing the energy 661.66keV}

The following picture is an illustration of a closer look at the photopeak of $^{60}$Co with the Gauschen fit.

\myImage[8cm]{img/na22_zoom}{The maximum of the fit is at channel (5642.27$\pm$5.66) containing the energy 1274.55keV}

After fitting the photopeaks of $^{60}$Co (has 2 ones), $^{137}$Cs and $^{22}$Na we plot the result in the following tabular and picture to illustrate the calibration of the NaJ-Scintillator-Detector.

\begin{center}
\begin{tabular}{c|c|c}
Channel & Energy & $\sigma$\\
\hline
(5379.24$\pm$1.85)& 1173.238 & 214.19\\
(6094.94$\pm$1.80) & 1332.513 &  193.32\\	
(3057.29$\pm$0.37) & 661.661 & 125.44\\
(5642.27$\pm$5.66) & 1274.545 & 197.15\\
\end{tabular}
\end{center}
\captionof{table}{Values examined from the plots for calculating picture 7 and the energy-resolution}

\myImage[8cm]{img/cal}{Results of the calibration of the NaJ-Scintillator}

Using these values we get an energy-resolution by using the following equation:

\begin{equation}
FWHM = 2\cdot \sqrt{2 ln 2} \cdot \sigma \cdot m
\end{equation}

Our $\sigma$ of $^{60}$Co is 193.32 and m =  0.225 so we get an resolution of 67.3keV. We need the m-factor in this equation in order to get FWHM in keV

\subsection{Absorptions coefficient of different materials}
To determine the absorption coefficient of different materials, we took a different number of plates of different materials. These materials had been Aluminium, Lead and Copper. The plates had have a thickness between 2 and 3 mm so in the first place we tokk a measurement without a plate to calibrate. After finishing this step, we proceeded with one plate until we had seven of them. We started with the Copper plates, then we measured the absorption of Aluminium and finished with lead.
In the following diagramms it is shown, how absorption is dependent of the material and its thickness.\\

\myImage[8cm]{img/al}{Result of our measurement for aborptions coefficient of Al}

The aborptions coefficient of Aluminium is:
\begin{center}
$\mu_{Al} = (0.0293 \pm 0.0039)\frac{1}{mm}$
\end{center}

\myImage[8cm]{img/cu}{Result of our measurement for aborptions coefficient of Cu}

So the aborptions coefficient of Copper is:
\begin{center}
$\mu_{Cu} = (0.0568 \pm 0.0083)\frac{1}{mm}$
\end{center}

\myImage[8cm]{img/pb}{Result of our measurement for aborptions coefficient of Pb}

And at least the aborptions coefficient of lead is:
\begin{center}
$\mu_{Pb} = (0.0698 \pm 0.0077)\frac{1}{mm}$
\end{center}

The values for $m$ are the activity at the beginning. $O$ represent the offset. As expected, the higher the atomic number of the material the higher is its aborptions coefficient.

\newpage
\subsection{1 Hour measurement of $^{60}$Co}
Before measureing $^{60}$Co for one houre, we had to switch some cables and turn the Ge-Detector on. To calibrate the Ge-detector we took again $^{60}$Co, $^{137}$Cs,$^{22}$Na and $^{241}$Am. We measured each of those elements for three minutes. The results of these measurements will help us to identify the peaks of the $^{60}$Co-diagramm. The following graphic represents our calibration-curve for the Ge-Detector.

\myImage[8cm]{img/calge}{Result of our calibration of the Ge-Detector}


Then we took the probe and measured the radiation for one hour. Our result is shown in the following graphic.

\myImage[11cm]{img/colong}{Result of our measurement of $^{60}$Co within 60 minutes}

The following tabular represents our results of the different peaks and their energy.
\\

\begin{tabular}{l|r|c|l}
& Energy [keV] & lit. value [keV] & \\
\hline
Backscatter-Peak & 215$\pm$20& 214.41	\\
Compton-Edge 1 & 960$\pm$20 & 963.38 	\\
Compton-Edge 2 & 1110$\pm$20 & 1118.06 	\\
Photo-Peak 1 & 1168$\pm$10 & 1173.21 	\\
Photo-Peak 2 & 1326$\pm$10 & 1332.47 	\\
\end{tabular}
\captionof{table}{Result of the Investigation of picture 12 (errors are valued because these peaks couldn't be fitted by Gaussian}

\subsection{Measurement of X-Ray conversion-lines}
For this experiment we took Barium and calibrated the Ge-Detector by measuring the probe for three minutes. Then we took four filters made of different materials. These materials were Tin (Sn), Antimon (Sb),  Tellurium (Tn) and Iodine (I).
