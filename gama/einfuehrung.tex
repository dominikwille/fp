\section{Ziele des Versuchs}
X-Ray-spectrometer are fundamental devices in nuclear physics because of the possibility of examining excited nuclear states. The objective of this experiment is now to examine the interaction between X-Rays and different materials, the reduction by absorbers and the evaluation of detectors by analysing different spectra.

\section{Theoretical backgound}

	\subsection{What is radiation}
	To answer this question and what kind of radiation it exists, we will take a closer look at the structure of a nucleus. A nucleus is made of three different kinds of particles. Electrons, protons and neutrons. By interaction of those particles it is possible that the nucleus gets unstable and fall apart by emitting energy (as photons) or parts of itself (particles). It depends on the strength of the radiation whether it's  $\alpha$- or $\beta$-particless or X-Rays. In the following part we will discuss those kinds of radiation closer.
\\

	\subsubsection*{$\alpha$-particles}
	$\alpha$-particles consists of two protons and two neutrons and are produced in the $\alpha$-decay. Beside this $^{4}$He-nucleus the mother-nucleus sets energy free. This energy can be calculated by the law $E=mc^{2}$. The decay can be described by:

	\begin{equation}
	^{A}_{Z}X \rightarrow ^{A-4}_{Z-2}Y + ^{4}_{2}He + \Delta E 
	\end{equation}

(A) stands for the mass number, (Z) means the atomic number, (X) is the decaing element und (Y) is the nucleus after the decay. The $\alpha$-particles are the waekest kind of radiation. A single paper is enougth to be proteced.

	\subsubsection*{$\beta$-particles}
There are two kinds of $\beta$-particles. The first one is $\beta^{+}$-decay where a proton gets emitted. Beside this proton, energy will be released. There is no law that describes the distribution of the energy. It is distributed as a value of zero to a characteristical maximum value on the three particles new nucleus, beta-particle and neutrino. 
Are there too many neutrons at tht core, we will have a $\beta^{-}$-decay where a neutron will be convert to a proton. After this conversion, the core releases an electron and an electron-antineutino. The following law describes this conversion
	\begin{equation}
	^{1}_{0}n \rightarrow ^{1}_{1}p + e^{-} + \overline{\nu_{e}}
	\end{equation}

In general it is:
	\begin{equation}
	^{A}_{Z}X \rightarrow ^{A}_{Z+1}Y + e^{-} + \overline{\nu_{e}}
	\end{equation}

Are there more protons in the nucleus, the $\beta^{+}$-decay begins. A proton gets converted into a neutron and the nucleus emits a positron and an elektron-neutrino. This process is described by:
	\begin{equation}
	^{1}_{1}p \rightarrow ^{1}_{0}n + e^{+} + \nu_{e}
	\end{equation}

also in general:
	\begin{equation}
	^{A}_{Z}X \rightarrow ^{A}_{Z-1}Y + e^{+} + \nu_{e}
	\end{equation}

To be protected you need various protective suits because the $\beta$-particles are stronger than the $\alpha$-particles.

	\subsubsection*{X-Rays}
	X-Rays in general are every radiation stronger than 200 keV or radiation with an unknown origin. At a closer look, it's electro-magnetic radiation produced by an $\alpha$- or $\beta$- particle being at an excited state. Descends this nucleus back to its groundstate it emits energy as X-Rays. Remarkably the nukleus only emits Photons and all other numbers of particles remain unchenged.
\\
	$\alpha$- and $\beta$- particles are charged particles, which interacts signifficantly more with matter as the Photons of the X-Rays do. This leads us to the result that X-Rays penetratets matter a lot stronger than the other particles do.

\subsection{Interaction of X-Rays with matter}

Meets a particle matter, there will be some various kinds of interactions between them, what we can measure in different ways. Physically, energy passes on and the matter gets warm. A chemical reaction for example would be the blackening of a photographic plate. An interaction of radiation with cells allways ends badly for the cell because of destruction.\\
It depends on the strength of the radiation which of those and the following cases of interaction will be observed.

\subsubsection*{Photo-Effekt}

If the energy of the photons is at least as big as the binding energy of a case electron the photon collides, so the $\gamma$-quant can knock out this elektron. The now free photoelektron has now the kinetic energy of:
	\begin{equation}
	E_{kin} = E_{\gamma}-E_{Bind}
	\end{equation}


Now is a place in the atomic shell free which hase to be occupied. An electron of a higher shell falls down and emits a characteristic X-Rays.\\
To calculate the probability of a scattering, it has to be distinguish between two cases. At the left hand there is a non-relativistic energy with $E_{\gamma}\ll m_{e}c_{0}^{2}$. The cross-section is:
	\begin{equation}
	\sigma_{Ph} \sim \frac{Z^{5}}{E_{\gamma}^{7/2}}
	\end{equation}
	
Considering the energy of $E_{\gamma} \gg m_{e}c_{0}^{2}$, it is:
	\begin{equation}
	\sigma_{Ph} \sim \frac{Z^{5}}{E_{\gamma}}
	\end{equation}
	
You can see clearly, that the largest cross-section for interaction is if the photon has a lower energy and the nuclues a higher atomic number

\subsubsection*{The Compton-Effekt}
The compton effect describes the adjustment of the wavelength of an elektron after a photon had scattered on it.

\myImage[8cm]{img/compt}{Scheme of the compton-effects}

So the wavelength of the photon extends throughout the loss of energy and it is:
	\begin{equation}
	\Delta \lambda = \lambda ' - \lambda
	\end{equation}

$\Delta \lambda$ is dependent on the angle $\phi$ and no more of the photoenergy
	\begin{equation}
	\Delta \lambda = \frac{h}{m_{e} c}(1-cos \phi)
	\end{equation}

After scattering, the angle-dependent energy of photons $E_{\nu}'$ and of electrons $E_{e}'$ gets described by the following equations:
	\begin{equation}
	E_{\nu}'(\phi) = \frac{E_{\nu}}{1+\frac{E_{\nu}}{m_{e}c^{2}}(1-cos 			\phi)}
	\end{equation}
and
	\begin{equation}
	E_{e}'(\phi) = E_{\nu}-E_{\nu}'(\phi) = E_{\nu}\left(1-\frac{1}{1+\frac{E_{\nu}}{m_{e}c^{2}}(1-cos \phi)}\right)
	\end{equation}

By scattering many photons you will receive a characteristical spectrum of energy. Through the fact, that every interaction transfers energy on the electrons, this specrtum is a function of the scattering angular $\phi$ , which is called the compton-continuum (as shown in picture 2).

\myImage[8cm]{img/konti}{Energy distrubution of the compton-elektrons}

Meets a photon with the angle 180 $ \ phi = ^ {\ circ} $ an electron, the maximum energy is transferred. This is shown in Figure 2 with the so-called Compton edge. This leads us to	
	\begin{equation}
	\label{kante}
	E_{e}'(180^{\circ}) = E_{\nu} \left(1-\frac{1}{1+\frac{2E_{\nu}}{m_{e}c^{2}}}\right) = \frac{2E_{\nu}^{2}}{m_{e}c^{2} +2E_{\nu}}
	\end{equation}

Additionally the spectrum contains a photopeak or $"$Full Energy Peak$"$. Equation \eqref{kante} shows that the compton-edge has to be on the left side of the associated photopeak.

The following picture is an example for a X-Ray In der Folgenden Abbildung wird ein Beispiel eines Gammaspektrums aufgeführt\\

\myImage[10cm]{img/bsp}{Gammaspectra with spectral lines at 4,4 MeV}

The cross-section of the compton-effekt gets calculated by the following equation:
	\begin{equation}
	\sigma_{C} \sim \frac{Z}{E_{\gamma}}
	\end{equation}
	
This effect mostly appears at elements with a small nuclear charge number and energy between 50 keV and 15 MeV or elements with a large  nuclear charge number and energy between 0,5 MeV and 5 MeV 
	
	\subsubsection*{Pairproduction and Annihilation}

Pairproduction appears if the energy of the photon is higher as twice the rest energy of an electron. The photon decays in the electric field of the nucleus into an electron and a positron. Due to the conservation rates, there has to exist a third impact-parameter. Mostly this parameter is the nucleus itself. For the cross-section follows now:
	\begin{equation}
	\sigma_{p} \sim Z^{2} ln E_{\gamma}
	\end{equation}

This effect appears on lightly elements with an energy bigger as 15 MeV and heavy elements with an energy bigger as 5 MeV.\\
\\
Annihilation is its reverse. Elementary particles and their antiparticle are getting converted into another particle. Due to this effect in the spectrum  appears an $"$Escape-lines$"$ 

	\subsubsection*{Photopeak and Escape-Peak}
	
As explained earlier, if an electron leaves an atom the gap has to be closed by an electron of an higher shell. This process emits X-Rays which are represented by the photopeak. If those X-Rays are leaving the detector, the spectrum contains an escape-peak. This peak simulates the existence of radioactive nucleid with the energy:
	\begin{equation}
	E_{esc} = E_{\gamma} - E_{e}
	\end{equation}

\subsection{NaJ-szintillator detector}
A NaJ-szintillator detector has at its front side a scintillator where the incomeing radiation causes free electrons due to the photo- and compton-effect. These primary electrons are creating secondary electrons by ionizing the atoms in a crystal which are emitting weak light flashes. These flashes are knocking electrons out of a photo cathode. The amount of these alectron can be measured by a secondary electron multiplier (SEM) (as shown in the following picture)

\myImage[10cm]{img/szint}{Schematic setup of a szintillation detektor}

\subsection{Ge-Semiconductor-Detector}
Semiconductor-detectors are detectors which are useing semiconductors to detect charged particles or photons. The radiation knocks electrons out of the valence-band into the conduction-band. To be sure that the conduction-band is as empty as possible, it gets refrigerate. Now there is an electric field, which causes, that the electrons and the holes are  travelling to the electrodes. If they the electrodes, they result an electric impuls which can be measured.