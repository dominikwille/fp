\section{Ziele des Versuchs}

In der Kernphysik sind $\gamma$-Spektrometer von grundlegender Bedeutung, da man mit ihnen Strahlung nachweisen kann und angeregte Kernzustände eingeordnet werden können. In diesem Versuch werden wir die Wechselwirkung von $\gamma$-Strahlung mit Materie, die Schwächung in Absorbern und die Energieauflösung der Detektoren durch die Analyse verschiedener Spektren untersuchen.

\section{Physikalische Grundlagen}

\subsection{Was ist Strahlung}
Um erklären zu können was Strahlung ist und was für Arten der Strahlung es gibt, sehen wir uns einmal den Aufbau eines Atoms an. Ein Atom besteht aus drei Arten von Teilchen und zwar den Protonen, den Neutronen und den Elektronen. Durch Wechselwirkungsprozesse dieser Teilchen kann es instabile Zustände geben, in denen Elemente mit der Zeit zerfallen. Diese instabile Zustände können auch künstlich erzeugt werden indem man  Elemente erst einmal "aktiviert", um einen Zerfallsprozess in Gang zu setzten. Bei einem Zerfallsprozess werden Teilchen emittiert. Je nach dem was für Teilchen es sind, spricht man entweder von der $\alpha$-, $\beta$- oder $\gamma$-Strahlung. Diese drei Arten von Strahlung haben unterschiedliche Eigenschaften, welche man in der Kernphysik einerseits untersucht und andererseits anwendet um wiederum andere Phänomene zu erklären.

\subsubsection*{$\alpha$-Strahlung}
$\alpha$-Strahlung ist ionisierte Strahlung, welche bei einem radioaktiven $\alpha$-Zerfall auftritt. Zum einen sendet der Nuklid ein $^{4}$He-Atomkern aus, welcher aus zwei Protonen und zwei Neutronen besteht und zum anderen wird Energie frei welche nach der Formel $E=mc^{2}$ berechnet werden kann. Es gilt für den Zerfallsprozess die Gleichung:

\begin{equation}
^{A}_{Z}X \rightarrow ^{A-4}_{Z-2}Y + ^{4}_{2}He + \Delta E 
\end{equation}

wenn (A) die Massenzahl, (Z) die Ordnungszahl, (X) das zerfallende Element und (Y) der sogenannte Tochternuklid ist. Die $\alpha$-Strahlung ist die schwächste Art der Strahlung da schon ein Blatt Paper ausreicht, um sich effektiv vor der Strahlung zu schützen.

\subsubsection{$\beta$-Strahlung}
Bei dem $\beta$-Zerfall handelt es sich ebenfalls um eine ionisierende Strahlung. Man unterscheidet hier zwischen einem $\beta^{+}$- und einem $\beta^{-}$-Zerfall. Entweder wird ein Elektron ($\beta^{-}$-Zerfall) oder ein Positron ($\beta^{+}$-Zerfall) emittiert. Bei diesem Prozess wird ebenfalls Energie freigesetzt, die allerdings keiner diskreten Verteilung folgt. Sie ist von Null bis zu einem für den Kern charakteristischen Maximalwert (etwa 1 MeV) auf die drei Teilchen Tochterkern, Betateilchen und Neutrino aufgeteilt.

Bei einem Neutronenüberschuss im Kern des Atoms tritt ein $\beta^{-}$-Zerfall ein. Dabei wandelt sich ein Neutron in eine Proton um und es wird ein Elektron und ein Elektron-Antineutrino abgestrahlt. Der Prozess folgt der Gleichung:
\begin{equation}
^{1}_{0}n \rightarrow ^{1}_{1}p + e^{-} + \overline{\nu_{e}}
\end{equation}

Wobei allgemein gilt:
\begin{equation}
^{A}_{Z}X \rightarrow ^{A}_{Z+1}Y + e^{-} + \overline{\nu_{e}}
\end{equation}

Ist der Nuklid protonenreich, tritt der $\beta^{+}$-Zerfall ein. Ein Proton wird in ein Neutron umgewandelt und sendet dabei ein Positron und ein Elektron-Neutrino aus. Es gilt also:
\begin{equation}
^{1}_{1}p \rightarrow ^{1}_{0}n + e^{+} + \nu_{e}
\end{equation}

und Allgemein gilt wieder:
\begin{equation}
^{A}_{Z}X \rightarrow ^{A}_{Z-1}Y + e^{+} + \nu_{e}
\end{equation}

\subsubsection*{$\gamma$-Strahlung}


\subsection{Wechselwirkung von $\gamma$-Strahlung mit Materie}
\subsubsection*{Photo-Effekt}
\subsubsection*{Compton-Effekt}
\subsubsection*{Paarbildung und Annihilation}
\subsection{Impulshöhenanalyse}
\subsubsection*{Photopeak und Escape-Peak}
\subsection{Messmethoden und -Geräte}