\section{Ziele des Versuchs}

In der Kernphysik sind $\gamma$-Spektrometer von grundlegender Bedeutung, da man mit ihnen Strahlung nachweisen kann und angeregte Kernzustände eingeordnet werden können. In diesem Versuch werden wir die Wechselwirkung von $\gamma$-Strahlung mit Materie, die Schwächung in Absorbern und die Energieauflösung der Detektoren durch die Analyse verschiedener Spektren untersuchen.

\section{Physikalische Grundlagen}

	\subsection{Was ist Strahlung}
	Um erklären zu können was Strahlung ist und was für Arten der Strahlung es gibt, sehen wir uns einmal den Aufbau eines Atoms an. Ein Atom besteht aus drei Arten von Teilchen und zwar den Protonen, den Neutronen und den Elektronen. Durch Wechselwirkungsprozesse dieser Teilchen kann es instabile Zustände geben, in denen Elemente mit der Zeit zerfallen. Diese instabile Zustände können auch künstlich erzeugt werden indem man  Elemente erst einmal "aktiviert", um einen Zerfallsprozess in Gang zu setzten. Bei einem Zerfallsprozess werden Teilchen emittiert. Je nach dem was für Teilchen es sind und wie stark sie Materie durchdringen, spricht man entweder von der $\alpha$-, $\beta$- oder $\gamma$-Strahlung. 
\\

	\subsubsection*{$\alpha$-Strahlung}
	$\alpha$-Strahlung ist ionisierte Strahlung, welche bei einem radioaktiven $\alpha$-Zerfall auftritt. Zum einen sendet der Nuklid ein $^{4}$He-Atomkern aus, welcher aus zwei Protonen und zwei Neutronen besteht und zum anderen wird Energie frei welche nach der Formel $E=mc^{2}$ berechnet werden kann. Es gilt für den Zerfallsprozess die Gleichung:

	\begin{equation}
	^{A}_{Z}X \rightarrow ^{A-4}_{Z-2}Y + ^{4}_{2}He + \Delta E 
	\end{equation}

wenn (A) die Massenzahl, (Z) die Ordnungszahl, (X) das zerfallende Element und (Y) der sogenannte Tochternuklid ist. Die $\alpha$-Strahlung ist die schwächste Art der Strahlung da schon ein Blatt Paper ausreicht, um sich effektiv vor der Strahlung zu schützen.

	\subsubsection*{$\beta$-Strahlung}
	Bei dem $\beta$-Zerfall handelt es sich ebenfalls um eine ionisierende Strahlung. Man unterscheidet hier zwischen einem $\beta^{+}$- und einem $\beta^{-}$-Zerfall. Entweder wird ein Elektron ($\beta^{-}$-Zerfall) oder ein Positron ($\beta^{+}$-Zerfall) emittiert. Bei diesem Prozess wird ebenfalls Energie freigesetzt, die allerdings keiner diskreten Verteilung folgt. Sie ist von Null bis zu einem für den Kern charakteristischen Maximalwert (etwa 1 MeV) auf die drei Teilchen Tochterkern, Betateilchen und Neutrino aufgeteilt.

Bei einem Neutronenüberschuss im Kern des Atoms tritt ein $\beta^{-}$-Zerfall ein. Dabei wandelt sich ein Neutron in eine Proton um und es wird ein Elektron und ein Elektron-Antineutrino abgestrahlt. Der Prozess folgt der Gleichung:
	\begin{equation}
	^{1}_{0}n \rightarrow ^{1}_{1}p + e^{-} + \overline{\nu_{e}}
	\end{equation}

Wobei allgemein gilt:
	\begin{equation}
	^{A}_{Z}X \rightarrow ^{A}_{Z+1}Y + e^{-} + \overline{\nu_{e}}
	\end{equation}

Ist der Nuklid protonenreich, tritt der $\beta^{+}$-Zerfall ein. Ein Proton wird in ein Neutron umgewandelt und sendet dabei ein Positron und ein Elektron-Neutrino aus. Es gilt also:
	\begin{equation}
	^{1}_{1}p \rightarrow ^{1}_{0}n + e^{+} + \nu_{e}
	\end{equation}

und Allgemein gilt wieder:
	\begin{equation}
	^{A}_{Z}X \rightarrow ^{A}_{Z-1}Y + e^{+} + \nu_{e}
	\end{equation}

Um sich gegen die $\beta$-Strahlung zu schützen braucht man verschiedene Schutzanzüge und -Vorkehrungen, da sie stärker ist als die $\alpha$-Strahlung.

	\subsubsection*{$\gamma$-Strahlung}
	Als $\gamma$-Strahlung bezeichnet man im weitesten Sinne jede Art von Strahlung, deren Quantenenergie größer als 200 keV ist oder deren Entstehungsart unbekannt ist. Enger betrachtet handelt es sich um eine elektromagnetische Strahlung welche entsteht, wenn ein $\alpha$- oder $\beta$-Strahler nach dem Zerfall in einem angeregten Zustand befindet. Fällt dieser Tochterkern wieder in einen niedrigeren, oder gar in den Grundzustand, wird Energie in Form von $\gamma$-Strahlung abgestrahlt ($"$Gammazerfall$"$). Das Besondere dieses $"$Zerfalls$"$ ist, dass der Kern nur Energie in Form von Photonen abgibt und es sonst keine Änderung der Anzahl seiner Neutronen und Protonen gibt.\\
\\
	Bei $\alpha$- und $\beta$-Strahlung werden also geladene Teilchen freigesetzt, welche deutlich stärker mit Materie wechselwirken als die Photonen und Quanten der $\gamma$-Strahlung. Daraus folgt, dass $\gamma$-Strahlung Materie am stärksten durchdringt.

\subsection{Wechselwirkung von $\gamma$-Strahlung mit Materie}
Trifft eine Strahlung auf Materie kommt es zu einer Wechselwirkung, welche sich auf unterschiedliche Arten bemerkbar macht. Physikalisch betrachtet wird Energie übertragen und es kommt zur Erwärmung. Eine chemische Reaktion wäre zum Beispiel die Schwärzung einer Photoplatte. Trifft eine Strahlung auf eine Zelle wird diese im biologischen Sinne beschädigt.\\
Je nachdem wie stark die $\gamma$-Strahlung ist, kann es zu verschiedenen Arten der Wechselwirkung kommen.

\subsubsection*{Photo-Effekt}
Ist die Energie des Photons mindestens genauso groß wie die Bindungsenergie eines Hüllenelektrons auf welches es trifft, so kann das $\gamma$-Quant dieses Elektron herausschlagen und das Atom dadurch Ionisieren. Das herausgeschlagene Photoelektron hat nach diesem Stoß eine kinetische Energie von:
	\begin{equation}
	E_{kin} = E_{\gamma}-E_{Bind}
	\end{equation}

Die frei gewordene Position in der Atomhülle wird von einem Elektron aus einem höheren Orbital besetzt. Bei diesem Vorgang wird von dem entsprechenden Elektron eine charakteristische Röntgenstrahlung emittiert.\\
Um die Wahrscheinlichkeit eines Stoßes zu berechnen muss man zwischen zwei Fällt unterscheiden. Zum einen gibt es den nicht-relativistischen Energiebereich mit $E_{\gamma}\ll m_{e}c_{0}^{2}$. Für ist der Wirkungsquerschnitt:
	\begin{equation}
	\sigma_{Ph} \sim \frac{Z^{5}}{E_{\gamma}^{7/2}}
	\end{equation}
	
Betrachtete man den Energiebereich mit $E_{\gamma} \gg m_{e}c_{0}^{2}$, so gilt:
	\begin{equation}
	\sigma_{Ph} \sim \frac{Z^{5}}{E_{\gamma}}
	\end{equation}
	
Daraus ist klar zu erkennen, dass die größte Wahrscheinlichkeit für die Wechselwirkung besteht, wenn das Photon eine geringe Energie und das Atom eine hohe Kernladungszahl hat.

\subsubsection*{Compton-Effekt}
Der Compton-Effekt beschreibt die Änderung der Wellenlänge eines Elektrons, nachdem ein Photon an diesem streute. 

\myImage[8cm]{img/compt}{Schema des Compton-Effekts}

Dabei verlängert sich die Wellenlänge des Photons durch den Energieverlust und es gilt:
	\begin{equation}
	\Delta \lambda = \lambda ' - \lambda
	\end{equation}

$\Delta \lambda$ hängt dabei nur noch vom Winkel $\phi$ ab und nicht mehr von der ursprünglichen Photoenergie.
	\begin{equation}
	\Delta \lambda = \frac{h}{m_{e} c}(1-cos \phi)
	\end{equation}

Die winkelabhängige Energie des Photons $E_{\nu}'$ und des Elektrons $E_{e}'$ nach der Streuung wird durch die folgenden Gleichungen beschrieben:
	\begin{equation}
	E_{\nu}'(\phi) = \frac{E_{\nu}}{1+\frac{E_{\nu}}{m_{e}c^{2}}(1-cos 			\phi)}
	\end{equation}
und
	\begin{equation}
	E_{e}'(\phi) = E_{\nu}-E_{\nu}'(\phi) = E_{\nu}\left(1-\frac{1}{1+\frac{E_{\nu}}{m_{e}c^{2}}(1-cos \phi)}\right)
	\end{equation}

Streut man nun viele Photonen, so gibt das ein charakteristisches Energiespektrum der gestreuten Elektronen. Da bei jeder Wechselwirkung Energie auf die Elektronen übertragen werden, ist diese eine Funktion des Streuwinkels $\phi$ , welches das Compton-Kontinuum genannt wird (siehe Abb.2).

\myImage[8cm]{img/konti}{Energieverteilung der Compton-Elektronen}

Der Wirkungsquerschnitt des Compton-Effekts errechnet sich aus der Formel:
	\begin{equation}
	\sigma_{C} \sim \frac{Z}{E_{\gamma}}
	\end{equation}
	
Für Elemente mit kleiner Kernladungszahl und Energien zwischen 50 keV und 15 MeV bzw. hoher Kernladungszahl und Energien zwischen 0,5 MeV und 5 MeV ist dieser Effekt der am häufigsten auftretende Wechselwirkungsprozess.

\subsubsection*{Paarbildung und Annihilation}
\subsection{Impulshöhenanalyse}
\subsubsection*{Photopeak und Escape-Peak}
\subsection{Messmethoden und -Geräte}