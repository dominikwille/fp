\section{Ziele des Versuchs}
x-ray-spectrometer are fundamental devices in nuclear physics because of the possibility of examining excited nuclear states. The objective of this experiment is now to examine the interaction between x-rays and different materials, the reduction by absorbers and the evaluation of detectors by analysing different spectra.

\section{Theoretical backgound}

	\subsection{What is radiation}
	To answer this question and what kind of radiation it exists, we will take a closer look at the structure of a nucleus. A nucleus is made of three different kinds of particles. Electrons, protons and neutrons. By interaction of those particles it is possible that the nucleus gets unstable and fall apart by emitting energy (as photons) or parts of itself (particles). It depends on the strength of the radiation whether it's  $\alpha$- or $\beta$-particless or x-rays. In the following part we will discuss those kinds of radiation closer.
\\

	\subsubsection*{$\alpha$-particles}
	$\alpha$-particles consists of two protons and two neutrons and are produced in the $\alpha$-decay. Beside this $^{4}$He-nucleus the mother-nucleus sets energy free. This energy can be calculated by the law $E=mc^{2}$. The decay can be described by:

	\begin{equation}
	^{A}_{Z}X \rightarrow ^{A-4}_{Z-2}Y + ^{4}_{2}He + \Delta E 
	\end{equation}

(A) stands for the mass number, (Z) means the atomic number, (X) is the decaing element und (Y) is the nucleus after the decay. The $\alpha$-particles are the waekest kind of radiation. A single paper is enougth to be proteced.

	\subsubsection*{$\beta$-particles}
There are two kinds of $\beta$-particles. The first one is $\beta^{+}$-decay where a proton gets emitted. Beside this proton, energy will be released. There is no law that describes the distribution of the energy. It is distributed as a value of zero to a characteristical maximum value on the three particles new nucleus, beta-particle and neutrino. 
Are there too many neutrons at tht core, we will have a $\beta^{-}$-decay where a neutron will be convert to a proton. After this conversion, the core releases an electron and an electron-antineutino. The following law describes this conversion
	\begin{equation}
	^{1}_{0}n \rightarrow ^{1}_{1}p + e^{-} + \overline{\nu_{e}}
	\end{equation}

In general it is:
	\begin{equation}
	^{A}_{Z}X \rightarrow ^{A}_{Z+1}Y + e^{-} + \overline{\nu_{e}}
	\end{equation}

Are there more protons in the nucleus, the $\beta^{+}$-decay begins. A proton gets converted into a neutron and the nucleus emits a positron and an elektron-neutrino. This process is described by:
	\begin{equation}
	^{1}_{1}p \rightarrow ^{1}_{0}n + e^{+} + \nu_{e}
	\end{equation}

also in general:
	\begin{equation}
	^{A}_{Z}X \rightarrow ^{A}_{Z-1}Y + e^{+} + \nu_{e}
	\end{equation}

To be protected you need various protective suits because the $\beta$-particles are stronger than the $\alpha$-particles.

	\subsubsection*{X-rays}
	X-rays in general are every radiation stronger than 200 keV or radiation with an unknown origin. At a closer look, it's electro-magnetic radiation produced by an $\alpha$- or $\beta$- particle being at an excited state. Descends this nucleus back to its groundstate it emits energy as x-rays. Remarkably the nukleus only emits Photons and all other numbers of particles remain unchenged.
\\
	$\alpha$- and $\beta$- particles are charged particles, which interacts signifficantly more with matter as the Photons of the x-rays do. This leads us to the result that x-rays penetratets matter a lot stronger than the other particles do.

\subsection{Interaction of x-rays with matter}

Meets a particle matter, there will be some various kinds of interactions between them, what we can measure in different ways. Physically, energy passes on and the matter gets warm. A chemical reaction for example would be the blackening of a photographic plate. An interaction of radiation with cells allways ends badly for the cell because of destruction.\\
It depends on the strength of the radiation which of those and the following cases of interaction will be observed.

\subsubsection*{Photo-Effekt}

If the energy of the photons is at least as big as the binding energy of a case electron the photon collides, so the $\gamma$-quant can knock out this elektron. The now free photoelektron has now the kinetic energy of:
	\begin{equation}
	E_{kin} = E_{\gamma}-E_{Bind}
	\end{equation}


Now is a place in the atomic shell free which hase to be occupied. An electron of a higher shell falls down and emits a characteristic x-ray.\\
To calculate the probability of a scattering, it has to be distinguish between two cases. At the left hand there is a non-relativistic energy with $E_{\gamma}\ll m_{e}c_{0}^{2}$. The cross-section is:
	\begin{equation}
	\sigma_{Ph} \sim \frac{Z^{5}}{E_{\gamma}^{7/2}}
	\end{equation}
	
Considering the energy of $E_{\gamma} \gg m_{e}c_{0}^{2}$, it is:
	\begin{equation}
	\sigma_{Ph} \sim \frac{Z^{5}}{E_{\gamma}}
	\end{equation}
	
You can see clearly, that the largest cross-section for interaction is if the photon has a lower energy and the nuclues a higher atomic number

\subsubsection*{The Compton-Effekt}
The compton effect describes the adjustment of the wavelength of an elektron after a photon had scattered on it.

\myImage[8cm]{img/compt}{Schema des Compton-Effekts}

So the wavelength of the photon extends throughout the loss of energy and it is:
	\begin{equation}
	\Delta \lambda = \lambda ' - \lambda
	\end{equation}

$\Delta \lambda$ is dependent on the angle $\phi$ and no more of the photoenergy
	\begin{equation}
	\Delta \lambda = \frac{h}{m_{e} c}(1-cos \phi)
	\end{equation}

After scattering, the angle-dependent energy of photons $E_{\nu}'$ and of electrons $E_{e}'$ gets described by the following equations:
	\begin{equation}
	E_{\nu}'(\phi) = \frac{E_{\nu}}{1+\frac{E_{\nu}}{m_{e}c^{2}}(1-cos 			\phi)}
	\end{equation}
and
	\begin{equation}
	E_{e}'(\phi) = E_{\nu}-E_{\nu}'(\phi) = E_{\nu}\left(1-\frac{1}{1+\frac{E_{\nu}}{m_{e}c^{2}}(1-cos \phi)}\right)
	\end{equation}

By scattering many photons you will receive a characteristical spectrum of energy. Through the fact, that every interaction transfers energy on the electrons, this specrtum is a function of the scattering angular $\phi$ , which is called the compton-continuum (as shown in picture 2).

\myImage[8cm]{img/konti}{Energieverteilung der Compton-Elektronen}

Meets a photon with the angle 180 $ \ phi = ^ {\ circ} $ an electron, the maximum energy is transferred. This is shown in Figure 2 with the so-called Compton edge. This leads us to	
	\begin{equation}
	\label{kante}
	E_{e}'(180^{\circ}) = E_{\nu} \left(1-\frac{1}{1+\frac{2E_{\nu}}{m_{e}c^{2}}}\right) = \frac{2E_{\nu}^{2}}{m_{e}c^{2} +2E_{\nu}}
	\end{equation}

Im Energiespektrum erhält man zusätzlich noch einen Photopeak bzw. ein $"$Full Energy Peak$"$. Aus Gleichung \eqref{kante} erkennt man, dass sich die zu einem Photopeak dazugehörende Compton-Kante um:
	\begin{equation}
	\frac{E_{\nu}}{1+\frac{2E_{\nu}}{m_{e}c^{2}}}
	\end{equation}
links von diesem befinden muss. In der Folgenden Abbildung wird ein Beispiel eines Gammaspektrums aufgeführt\\

\myImage[10cm]{img/bsp}{Gammaspektrum mit Spektrallinie bei 4,4 MeV}

Der Wirkungsquerschnitt des Compton-Effekts errechnet sich aus der Formel:
	\begin{equation}
	\sigma_{C} \sim \frac{Z}{E_{\gamma}}
	\end{equation}
	
Für Elemente mit kleiner Kernladungszahl und Energien zwischen 50 keV und 15 MeV bzw. hoher Kernladungszahl und Energien zwischen 0,5 MeV und 5 MeV ist dieser Effekt der am häufigsten auftretende Wechselwirkungsprozess.

	\subsubsection*{Paarbildung und Annihilation}
	Der Effekt der Paarbildung tritt auf, wenn die Energie des Photons größer ist als die doppelte Ruheenergie eines Elektrons. Dann zerfällt das Photon im elektrischen Feld des Atomkerns in ein Elektron und ein Positron. Auch hier bleiben die Erhaltungssätze bestehen, wodurch ein dritter Stoßpartner benötigt wird. In den meisten Fällen handelt es sich dabei um den Atomkern, was zur Folge hat, dass für den Wirkungsquerschnitt gilt:
	\begin{equation}
	\sigma_{p} \sim Z^{2} ln E_{\gamma}
	\end{equation}

Dieser Effekt tritt bei leichten Elementen mit Energien größer als 15 MeV und schweren Elementen mit Energien größer als 5 MeV auf.\\
\\
Bei der Annihilation, bzw. der Paarvernichtung handelt es sich um den Umgekehrten Effekt, bei dem ein Elementarteilchen und sein Antiteilchen sich zusammen in ein anderes Teilchen verwandelt. Dabei treten sogenannte $"$Escape-Linien$"$ im Spektrum auf. 

	\subsubsection*{Photopeak und Escape-Peak}
	Wenn Elektronen durch den Photoeffekt aus der Atomschale herausgelöst worden sind, müssen die Lücken wie oben beschrieben, wieder geschlossen werden. Bei diesem Vorgang wird Röntgenstrahlung emittiert, welche in dem Energiespektrum als Photopeak aufgezeichnet werden. Entweicht diese Röntgenstrahlung allerdings aus dem Detektor, so kommt es zur Ausbildung eines Escape-Peaks. Es werden in dem aufgezeichneten Spektrum nicht vorhandene radioaktive Nuklide vorgetäuscht mit der Energie:
	\begin{equation}
	E_{esc} = E_{\gamma} - E_{e}
	\end{equation}

\subsection{Szintillationsdetektor}
Ein Szintillationszähler besitzt an seinen vorderen Ende ein Szintillaot in welchem durch einfallende Strahlung Lichtblitze erzeugt werden. Die Anzahl dieser Lichtblitze hängt von der Energie der einfallenden Strahlung ab. Durch den Photoeffekt werden aus dem dahinter sitzten Photomultiplier Elektronen frei. Durch Stöße an den Elektroden setzt sich eine Lawine in Bewegung, die an der Anode gemessen wird. Die Größe der Amplitude gibt an, wie stark die einfallende Strahlung ist. In der folgenden Abbildung wird das nochmals schematisch dargestellt.

\myImage[10cm]{img/szint}{Schematischer Aufbau eines Szintillationsdetektor}