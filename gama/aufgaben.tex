\section{Aufgaben}
\subsection{Aufgabe 1}
Inbetriebnahme des NaJ-Detektors (Hochspannung wird vom Assistenten eingestellt). Betrachten der Impulsformen ($^{137}$Cs-Quelle) vor und hinter dem Hauptverstärker; Skizze der Impulsformen ins Protokoll (Achsenbeschriftung!); Bestimmung der Anstiegs-und Abfallzeiten der Impulse; Diskussion.

\subsection{Aufgabe 2}
Energieeichung des NaJ-Detektors mit $^{60}$Co, $^{22}$Na, $^{137}$Cs und Bestimmung der Energieauflösung für alle Photopeaks. Dabei ist die Verstärkung am Hauptverstärker so einzustellen, dass der Photopeak der 1.33 MeV $^{60}$Co-Linie im Konversionsbereich des VKA liegt. Bestimmung der Peaklagen und Halbwertsbreiten direkt am Rechner (Cursor, Fehler 
abschätzen!). Messpunkte und Eichgerade auf Millimeterpapier darstellen. 

\subsection{Aufgabe 3}
Bestimmung des Massenschwächungskoeffizienten von Pb, Cu und Al für die 1.33 MeV $^{60}$Co-Linie. Hierzu die starke $^{60}$Co-Quelle (wird vom Assistenten eingesetzt), Kollimator und NaJ-Detektor verwenden. Die Zählrate wird durch Integration über den Photopeak direkt am Rechner bestimmt (Festlegen einer Region of Interest(ROI), Zeitvorwahl, Totzeit 
beachten!). Graphische Darstellung der Zählrate über der Absorberdicke parallel zur Messung. 

\subsection{Aufgabe 4}
Inbetriebnahme des Ge-Detektors (Hochspannung wird vom Assistenten eingestellt). Protokollieren der Impulsformen vor und nach dem Hauptverstärker, Bestimmung der Anstiegs- und Abfallzeiten der Impulse.

\subsection{Aufgabe 5}
Energieeichung des Ge-Detektors mit $^{60}$Co (Verstärkung wie in 2. abgleichen), $^{22}$Na, $^{137}$Cs, $^{241}$Am, und Bestimmung der Energieauflösung für alle Photopeaks. Aufnahme eines $^{60}$Co-Spektrums mit guter Statistik, grafische Ausgabe und quantitative Diskussion 
(Bestimmung von Compton-Kanten, "Back-Scatter"-Linien, “Escape"-Linien; Vergleich mit berechneten Werten). 

\subsection{Aufgabe 6}
Bestimmung der Röntgenkonversionsline von $^{133}$Ba (E ~ 30 keV) über die kritische Absorption (Ge-Detektor). Als Absorber sind Sn, Sb, Te und J verfügbar. Um Einflüsse der Geometrie und der Absorberdicke auf die absolute Zählrate zu kontrollieren und zu eliminieren, sollte die 81-keV-Linie als Referenz genutzt werden. Die Messgrößen (Zählrate im interessierenden Peak bzw. Verhältnis dieser Zählrate zur Referenzzählrate) sind grafisch über der Lage der Absorptionskante für den entsprechenden Absorber darzustellen.