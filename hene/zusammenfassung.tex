\section{Zusammenfassung und Diskussion}

\subsection{Aufgabe 1}
Während des Aufbaus und der Justage probierten wir an den einzelnen Komponenten, wie sich was verändert, wenn man zum Beispiel die Spiegel S1 und S2 vertauscht. Alles in allem verlief die Justage aber immer recht zügig und ohne Probleme.

\subsection{Aufgabe 2}
Die Ausgangsleistung des Justierlasers betrug $P_0 = 2$ mW. Wir erhielten eine Leistung von $P = (0,50 \pm 0,05)\mu W$. Das entspricht einem Viertel der ursprünglichen Leistung. Aufgrund des Alters des Lasers verflüchtigt sich das Helium und das Neon. Daraus folgt, dass die Konzentration der benötigten Gase sinkt und damit auch die Laserleistung.

\subsection{Aufgabe 3}
Wir errechneten eine Resonatorlänge von $d = (23,44 \pm 0,37)$ cm, was durchaus ein realistischer Wert ist. Messen konnten wir die Resonatorlänge nicht, weil der Justierlaser ein Gehäuse besitzt, welches natürlich größer als der Resonator selbst ist.

\subsection{Aufgabe 5}
Hier sollten wir die Verstärkung der Plasmaröhre berechnen und erhielten einen Wert von $(6,25 \pm 0,73)\%$. Auch dieser Wert erachten wir als realistisch. Vergleichswerte konnten wir nur von einer anderen Gruppe erhalten, welche ein vergleichbares Ergebnis erhielt.

Wie schon in der Aufgabe selber beschrieben, übertrifft diese Verstärkung die Verluste, welche durch die Spiegel verursacht werden, bei weitem.

\subsection{Aufgabe 6}
Um diese Aufgabe zu bewältigen, bewegten wir den Spiegel S1 auf der Schiene hin und her, um verschiedene Abstände, bei welchen die Verstärkungen auftreten, zu erfassen. Anschließend ruckelten wir an dem Spiegel bei verschiedenen Abständen. Dabei erkannten wir, dass durch seitliches Auslenken noch mehr Positionen erfassen konnte, bei denen die Verstärkungen auftraten.

\subsection{Aufgabe 7}
Für die maximale Resonatorlänge erhielten wir einen Wert von $d = (51,0\pm 0,5)$ cm. Größer konnte der Resonator nicht mehr werden, was wohl durch die Streueigenschaften von Licht verursacht wird. Ab einer gewissen Distanz kann der Hohlspiegel nicht mehr genügen Photonen erfassen und reflektieren, damit es in der Plasmaröhre zu weiteren Emissionen kommt.

\subsection{Aufgabe 8}
Diese Aufgabe konnten wir erfolgreich bearbeiten, was anhand des Bildes klar erkennbar ist.

\subsection{Aufgabe 9}
Bei dieser Aufgabe taten sich so manche Schwierigkeiten auf mit dem richtigen erfassen von Bildern. Für die Auswertung haben wir deswegen nur 3 von 5 Bildern benutzt, bei denen es eindeutig war, welche Moden zu sehen waren. Trotzdem haben wir bei allen Ergebnissen einen Faktor 2 welcher uns bei den theoretischen Werten fehlt.

\begin{center}
\begin{tabular}{c | c | c}
Abstand [cm] & $\Delta \nu_{exp}$ [GHz] & $\Delta \nu_{theo}$ [GHz]\\ \hline
46,8 $\pm$ 1,0 & 0,64 $\pm$ 0,2 & 0,32 $\pm$ 0,03\\ 
48,3 $\pm$ 1,0 & 0,60 $\pm$ 0,2 & 0,31 $\pm$ 0,03\\
49,6 $\pm$ 1,0 & 0,62 $\pm$ 0,2 & 0,30 $\pm$ 0,03\\
\end{tabular}
\end{center}

Ein möglicher Grund könnte sein, dass aus Versehen die Einstellungen des Oszillators verändert wurden und somit das freie Spektrum nicht mehr 2 GHz sondern nur noch 1 GHz betrug. 

\subsection{Aufgabe 10}

\subsection{Aufgabe 11}
Wie an dem Bild klar erkennbar ist, haben wir die Power Dips gefunden und können somit sagen, dass diese Aufgabe ebenfalls erfolgreich abgeschlossen wurde.