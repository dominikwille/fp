\subsection{Aufgabe 2}

Die Messung der Justierleserleistung wurde von uns durch den vorgegebenen Bandpassfilter von 633 nm durchgeführt. Dabei wurde das Messgerät '...' verwendet. Die Fehler wurden auf $\pm$ 50 $\mu$W geschätzt, da keine Beschreibung des Messgeräts zu finden war. Die gemessene Leistung betrug:

\begin{equation}
P = (500 \pm 50) \mu W
\end{equation}

Die Messung der Transmission der Spiegel haben wir für jeden Spiegel getrennt durchgeführt. Die Messungen ergaben folgende Werte:

- Für den planaren Spiegel S2:

\begin{equation}
I_{2} = (0.50 \pm 0.05) \mu W
\end{equation}

- Für den Hohlspiegel S1:

\begin{equation}
	I_{2} = (0.55 \pm 0.05) \mu W
\end{equation}

Wie man sieht, entsprechen die Werte den erwarteten Werten von $\approx$ 0,1\%. Die Transmission wird durch 

\begin{align}
	T &= \frac{I_{T}}{I_{0}} \\
	T_{1} &= (0.11 \pm 0.015)\% \\
	T_{2} &= (0.10 \pm 0.014)\% 
\end{align}

Die Fehler wurden mit Hilfe der Gaußsche Fehlerfortpflanzung ausgerechnet.

Die Polarisationsmessung war sehr ungenau. Wir hatten keine Möglichkeit den Polarizationsfilter aufzustellen, mussten den Filter in der Hand so ruhig wie möglich halten und die Messung durchführen. Unsere Messung ergab ein Maximum der Intensität bei $(1 \pm 2)^{\circ}$.

%%% TeX-master: ''hene.tex''
