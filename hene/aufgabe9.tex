\subsection{Aufgabe 9}

Anfangs hatten wir ein paar Schwierigkeiten mit dem herausfinden der Moden, welche wir für diese Aufgabe brauchen. Doch als wir die Spiegel ein wenig bewegten, konnten wir ein paar Moden erzeugen.
Für die Abstände und die theoretischen bzw. experimentellen Frequenzen bekamen wir folgende Werte:

\begin{center}
\begin{tabular}{c | c | c}
Abstand [cm] & $\Delta \nu_{exp}$ [GHz] & $\Delta \nu_{theo}$ [GHz]\\ \hline
46,8 $\pm$ 1,0 & 0,64 $\pm$ 0,2 & 0,32 $\pm$ 0,03 \\ 
48,3 $\pm$ 1,0 & 0,60 $\pm$ 0,2 & 0,31 $\pm$ 0,03\\
49,6 $\pm$ 1,0 & 0,62 $\pm$ 0,2 & 0,30 $\pm$ 0,03\\
\end{tabular}
\end{center}

Die Theoretischen werte errechnen sich aus Gl. 9. Wir wählten die Fehler der Abstände auf 1cm, weil es relativ schwer war den genauen Standort der Spiegel und somit deren Distanz zu ermitteln. Der Fehlerwert der experimentell bestimmten Frequenzen unterliegt der Ablesegenauigkeit und der Fehler des theoretischen Wertes beträgt 10$\%$ des Wertes selbst.