\subsection{Aufgabe 3}
Die Modenstruktur des Justierlasers kann dafür herangezogen werden um die ideale Resonatorlänge zu bestimmen. Es gilt:
\begin{align}
k &= \frac{2\pi}{\lambda}  = \frac{n\pi}{d} \\
\Rightarrow \Delta f &= \frac{c}{\lambda} = \frac{c}{2d} \\
\Rightarrow d &= \frac{c}{2\Delta f}
\end{align}
\myImage[10cm]{img/55}{Modenspecktrum des Justierlasers}
Gemessen wurde bei einer Frequenz von $f = 2\, \text{GHz}$, wie zu sehen ist haben die moden eine Frequenzdifferenz von $(640 \pm 10)\, \text{MHz}$. Das führt zu einer idealen Resonatorlänge von:
\begin{equation}
d = (23,44 \pm 0,37)\, \text{cm}
\end{equation}
%%% TeX-master: "hene.tex"
