\subsection{Aufgabe 1}

Als erstes nahmen wir den Justierlaser in Betrieb und stellten die Spiegel S1, S3 und S4 an ihre Positionen. Dabei musste die Distanz zwischen dem Spiegel S1 und der späteren Position des Spiegels S2 so groß sein, dass für das Aktive Medium genug Platz vorhanden war. Das Aktive Medium wurde allerdings erst später eingefügt.
Anschließend führten wir mithilfe des Spiegels S4 den Laser direkt zurück zu seiner Quelle. Nun setzten wir den Spiegel S2 auf die Schiene zwischen Spiegel S1 und S3. Dadurch veränderte sich erneut die Rückführung des Lasers. Wir stellten also wieder mit dem Spiegel S4 den Laser so, dass der Strahl zurück zur Quelle gelangt. Bei einer weiteren Überprüfung bemerkten wir, dass der Laserstrahl nicht mittig auf den Spiegel S1 fiel. Dies Korrigierten wir durch eine erneute Justierung des Spiegels S3. Dadurch musste aber auch wieder der Spiegel S4 nachjustiert werden, bis wir letzten Endes mit dem Strahlengang zufrieden waren.