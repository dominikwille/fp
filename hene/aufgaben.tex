\section{Aufgaben}

\subsection*{Aufgabe 1}
Ausrichtung des Justierlaserstrahls entlang der optischen Bank mit Hilfe der Umlenkspiegel. Der Strahl muss mittig durch die hintere Öffnung des Resonatorplanspiegels bei Aufstellung an beiden Enden der optischen Bank gehen.

\subsection*{Aufgabe 2}
Messung der Justierleistung und -polarisation und der Transmission der später für den Resonator verwendeten Spiegel.

\subsection*{Aufgabe 3}
Messung der Modenstruktur des Justierlasers und darüber Bestimmung der Resonatorlänge.

\subsection*{Aufgabe 4}
Aufbau und Justage des passiven Resonators, Beobachtung der Moden auf dem Schirm Grobbestimmung der Radien über Stabilitätskriterium. Warum flackern die Moden bei stillstehenden Spiegeln? Warum nicht bei bewegten Spiegeln?

\subsection*{Aufgabe 5}
Justage der Plasmaröhre ohne Resonator und Messung der Laserverstärkung \(\frac{\Delta I}{I}\)\ (Messung von \(\Delta I\) bei Ein- oder Ausschalten der Gasentladung). Hierbei wird das 
Leuchten der Gasentladung durch den Bandpassfilter unterdrückt (der Filter ist um die 633 nm-Linie zentriert). Überprüfen Sie, ob die Unterdrückung ausreichend ist. 
Diskutieren Sie das Ergebnis im Vergleich mit dem von Aufgabenteil 2. 

\subsection*{Aufgabe 6}
Inbetriebnahme und Optimierung des Messlasers. Messung der Ausgangsleistung und der Polarisation (Diskussion!). 

\subsection*{Aufgabe 7}
Bestimmung der Spiegelradien über das Stabilitätskriterium.

\subsection*{Aufgabe 8}
Beobachtung der Moden auf dem Schirm und mit dem Fabry-Perlot-Interferometer. Unterdrückung höherer transversaler Moden mit Hilfe der Irisblende.

\subsection*{Aufgabe 9}
Messung des Abstandes der Longitudinalmodenfrequenzen  in Abhängigkeit vom Spiegelstand.

\subsection*{Aufgabe 10}
Messung des kompletten Modenspektrums (ein \("\)Schuss\("\) mit dem digitalen Oszilloskop) bei offener Irisblende für einen der unter 9 verwendeten Spiegelabstände. Versuchen sie eine Zuordnung der gemessenen Linie gemäß der obigen Formel für die Modenfrequenzen des stabilen Resonators.

\subsection*{Aufgabe 11}
Bestimmung der Linienbreite und der Plasmatemperatur (Diskussion!). Eventuelle Beobachtung des power dips.