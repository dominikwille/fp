\subsection{Aufgabe 5}

Für diese Aufgabe entfernten wir die Spiegel S1 und S2 und befestigten die Plasmaröhre über der Schiene. Hinter der Plasmaröhre platzierten wir den Bandpassfilter und den Photosensor des Messgeräts. Wir schalteten anschließend den Justierlaser wieder ein und drehten so lange an den Manipulaturen der Plasmagsröhre, bis wir mit dem Messgerät eine möglichst große Spannung wahrnahmen. Damit wussten wir, dass der Laserstrahl möglichst gerade durch die Röhre verlief.\\
Anschließend schalteten wir die Röhre an und notierten uns den angezeigten Wert.\\

\begin{center}
\begin{tabular}{c | c}
Stand Lampe & Stromstärke I\\
\hline
Lampe an & $I_1$ = (59,2 $\pm$ 5) $\mu$W\\
Lampe aus & $I_2$ = (62,9 $\pm$ 5) $\mu$W

\end{tabular}
\end{center}

Unter der Anwendung der Formel:

\begin{equation}
\frac{\Delta I}{I} = I_{verstärkt}
\end{equation}

mit $\Delta I$ = $I_2$ - $I_1$ = 3,7$\mu$W entspricht dies einer Verstärkung um (6,25$\pm$ 0,73)$\%$. Wir wählten die Fehlerwerte so groß, weil es relativ schwierig war von Hand den Photosensor so zu halten, dass der Laser genau traf.\\
Auf dem Interferometer war keine Änderung während des Ein- und Ausschaltvorgangs erkennbar. Der Bandpassfilter hat das Leuchten also ausreichend unterdrückt.\\
Unsere Ergebnisse aus Aufgabe 2 waren:

\begin{align*}
	T_{1} &= (0.11 \pm 0.015)\% \\
	T_{2} &= (0.10 \pm 0.014)\% 
\end{align*}

Somit ist klar erkennbar, dass die Verstärkung von (6,25$\pm$ 0,73)$\%$ mehr als Genug ist um die Verluste der Spiegel auszugleichen. 