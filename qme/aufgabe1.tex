\subsection{Aufgabe 1 - Einschalten des Versuchsaufbaus}
Die Geräte wurde von unserem Versuchsleiter eingeschaltet und er wies uns in die Funktionsweise der Geräte ein. Danach stellten wir die Spannungen $U_{FR}$, $U_{KA}$ und $U_{FA}$ auf die folgenden Werte ein:

\begin{tabular}{l|l|l}
$U_{FR}$ & = $113\pm 2$ V & (Beschleunigungsspannung)\\
$U_{KA}$ & = 0V & (Glühkathode)\\
$U_{FA}$ & = $106\pm 0,5$V & (Massenfilters)\\
\end{tabular}

Diese Werte wurden während der folgenden Messung nicht mehr verändert. 

\subsection{Restgas-Messung}
Bei dieser Messung machten wir uns mit dem Programm LabView vertraut und stellten die Verstärkung auf $10^{-10}$. Der Druck betrug dabei $8,9 \cdot 10^{-6}$ mbar. Zudem schlossen wir Ventil V3, öffneten das Ventil V4 und nahmen mehrere Messungen von dem Restgas. In der folgenden Abbildung wurden die gemittelten Ergebnisse dieser Messungen graphisch dargestellt.\\
Aus [4] wissen wir, dass der Peak bei m/q=18 für den Wasseranteil in dem Gas steht. Das liegt daran, dass in dem Rezipienten immer ein geringer Wasseranteil vorhanden ist, der sich, wenn das Gerät ausgeschaltet ist, auf der Glühkathode niederschlägt.

\myImage[9cm]{img/RGbalk}{Überarbeitete Daten aus der Restgasmessung}

\newpage
\subsection{Testgas 1 Dosierung und Messung}
Über das Ventil V1A-D wurde das Testgas (in diesem Fall Argon) die Vorkammer gelassen bis die Nadel am Druckmessgerät einmal kurz ausschlug. Anschließend öffneten wir das Ventil V2 und das Dosierventil vorsichtig, damit das Gas in den Rezipienten gelangen konnte. Nun stellten wir den Emissionsstrom der Glühkathode auf 0,2 mA, warteten kurz und notierten den Druck im Rezipienten. Anschließend starteten wir am PC das Messprogramm. Als dieses fertig war, schlossen wir das Ventil V4 und Dosierventil und öffneten vorsichtig das Ventil V3. Wir warteten anschließend 10 Minuten, bis die Vorpumpe den Rezipienten wieder leerte. Diesen Vorgang wiederholten wir für die Messung von Testgas 2 und 3.\\
Um die Messdaten auszuwerten, änderten wir die Vorzeichen der Häufigkeiten und stellten die Graphen durch Balken dar.