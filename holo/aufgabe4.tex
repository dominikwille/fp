\subsection*{Aufgabe 4 - Die Erstellung der Hologramme}
Nach der Einweisung unsres Tutors bauten wir aus weißem Papier ein Modell eines Fliegers und eines griechischen Tempels. 
Um mit den Fotoplatten arbeiten zu können ohne dass sie zerstört werden,  verdunkelten wir den Raum und schalteten eine Grünlichtlampe an. Diese Sorte von Lampen kommt bei der Arbeit mit Fotoplatten zum Einsatz, weil diese mit rotem Licht reagieren und so kann es passieren, dass sie unbrauchbar werden.\\
Da man auf einer Fotoplatte zwei Hologramme aufnehmen konnte, entschieden wir uns, sie in der Mitte auseinander zu schneiden. Die eine Seite verwendeten wir für Hologramme mit der Belichtungszeit von einer Sekunde. Die andere Seite sollte einer Belichtungszeit von zwei Sekunden ausgesetzt werden und wurde durch eine abgeschnittene Ecke gekennzeichnet.\\
Nun platzierten wir die Fotoplatte in der vorgesehenen Halterung und den Mini-Papierflieger auf einem schwarzen Untergrund hinter der Fotoplatte. Wir schlossen den Deckel des Gehäuses und betätigten einen Schalter, der die Beleuchtungszeit kontrollierte.\\
Anschließend entnahmen wir die Fotoplatte aus der Halterung und ersetzten sie durch eine neue. Wir stellten den Timer für die Belichtungszeit auf 2 Sekunden und drückten wieder auf den Auslöser. Diese Prozedur wiederholten wir für alle 6 Hologramme, wobei wir uns am ende entschieden, dass wir 3 Hologramme des Tempels aufnehmen. Zwei dieser Aufnahmen hatten eine Beleuchtungszeit von je einer Sekunde, während das dritte eine von zwei Sekunden hatte.\\
Anschließend verpackten wir die Fotoplatten ein einer Box um sie vor ungewolltem Lichteinfall zu schützen und brachten sie ins Labor zur Entwicklung.