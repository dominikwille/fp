\subsection*{Aufgabe 5 - Die Entwicklung unserer Hologramme}
Im Labor angekommen, zogen sich zwei von uns einen weißen Laborkittel und Plastikhandschuhe an, was uns vor den verwendeten Chemikalien schützen sollte. Da die Fotoplatten immer noch durch ungewollte Lichteinflüsse zerstört werden konnten, arbeiteten wir in dem Labor wieder bei dem grünen Licht. Zur Entwicklung des belichteten Filmmaterials haben wir dieses für zwei Minuten in eine Fixierlösung aus Ascorbinsäure und Natriumphosphat gelegt. Die Fotoplatten hatten sich darin schwarz verfärbt.
Nach kurzem Auswaschen in destilliertem Wasser haben wir es dann etwa zwei Minuten in ein Bleichbad aus Chromschwefelsäure gelegt, welches aus dem durch das Fixieren entstandene Amplitudenhologramm wieder ein Phasenhologramm erzeugt hat. In dieser Lösung wurden die Fotoplatten wieder durchsichtig. Als letztes legten wir das Filmmaterial für etwa 40 Minuten in destilliertes Wasser. Anschließend hingen wir die Bilder für eine weitere Stunde in in den Trockenschrank. Bei den fertigen Hologrammen war unser Objekt wie erwartet - durch das Trocknen und den damit verbundenen Schrumpfungsprozess grün schimmernd - dreidimensional zu sehen. Insgesamt sind vier von sechs Hologrammen erfolgreich erstellt worden.