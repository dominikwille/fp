\subsection*{Aufgabe 5 - Die Entwicklung unserer Hologramme}
Zur Entwicklung des belichteten Filmmaterials haben wir dieses für zwei Minuten in eine Fixierlösung aus Ascorbinsäure und Natriumphosphat gelegt. Nach kurzem Auswaschen haben wir es dann in ein Bleichbad aus Chromschwefelsäure gelegt, welches aus dem durch das Fixieren entstandene Amplitudenhologramm wieder ein Phasenhologramm erzeugt hat. Bei den fertigen Hologrammen war unser Objekt wie erwartet - durch das Trocknen und den damit verbundenen Schrumpfungsprozess grün schimmernd - dreidimensional zu sehen.