\subsection*{Aufgabe 5 - Die Entwicklung unserer Hologramme}
Zur Entwicklung des belichteten Filmmaterials haben wir dieses für zwei Minuten in eine Fixierlösung aus Ascorbinsäure und Natriumphosphat gelegt. Nach kurzem Auswaschen in destilliertem Wasser haben wir es dann für etwa 40 Minuten in ein Bleichbad aus Chromschwefelsäure gelegt, welches aus dem durch das Fixieren entstandene Amplitudenhologramm wieder ein Phasenhologramm erzeugt hat. Anschließend hingen wir die Bilder für eine weitere Stunde in in den Trockner. Bei den fertigen Hologrammen war unser Objekt wie erwartet - durch das Trocknen und den damit verbundenen Schrumpfungsprozess grün schimmernd - dreidimensional zu sehen.