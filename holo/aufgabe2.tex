\subsection*{Aufgabe 2 - Das Michelson-Interferometer}
Um diese Aufgabe zu lösen, bauten wir aus den gegebenen Materialien wie verlangt, ein Michelson-Interferometer. Bei dem Aufbau selbst gab es keine weiteren Schwierigkeiten. 
Als erstes Schalteten wir den Laser an und platzierten einen Spiegel ihm gegenüber. Wir justierten den Spiegel so, dass der Strahl direkt zurück geworfen wurde. Der Spiegel konnte mittels eines magnetischen Fußes auf dem Tisch fixiert werden. Anschließend platzierten wir den Strahlteiler im Winkel von etwa $45^{\circ}$ in der Mitte zwischen Laser und Spiegel. Auch diese Komponente wurde mithilfe eines Magneten fixiert.
Nun positionierten wir den zweiten Spiegel so auf dem Tisch, dass der durch den Strahlteiler aufgespaltene Laserstrahl auch diesen Spiegel erreicht und von ihm reflektiert wird. Wie auch beim ersten Spiegel, soll der zweite Spiegel den Laser über den Strahlteiler zurück zur Quelle reflektieren.
Direkt gegenüber des zweiten Spiegels platzierten wir eine weiße Leinwand, auf welcher die Teilstrahlen des Lasers fielen. Nun justierten wir die Spiegel in Position und Stellung nach, sodass beide Teilstrahlen zum einen direkt zurück in die Quelle reflektiert wurden und zum anderen auf einen Punkt auf der Leinwand fielen. 
Die Leinwand wurde später natürlich durch dem Photosensor des Messgeräts ersetzt.
Der Aufbau sah letzten Endes genauso aus, wie in Abb.2 schematisch und in Abb. 3 fotografisch dargestellt wurde.



Um die Stabilität des Aufbaus zu untersuchen, machten wir eine Aufnahme von dem eingeschalteten Laser und verschiedenen Geräuschen. Als erstes ist es ganz ruhig gewesen. Anschließend sprachen wir miteinander. Nach einer Pause liefen wir durch den Raum, pausierten, wackelten an der Box, pausierten wieder und sprangen in die Luft (An diesem Punkt bearbeiten wir noch gleichzeitig die Aufgabe 11) .
Als Ergebnis sah unsere Aufnahme wie folgt aus: