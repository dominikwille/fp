\section{Vorwort}
Wenn man ein Photo eines Gegenstands aufnimmt, so wird mittels inkohärentem Licht die Information auf eine Photoplatte bzw. einem Chip gespeichert. Dabei gehen allerdings die Tiefeninformationen des Objekts verloren, weil nur die Amplitude des Lichts gespeichert wird. Das Ergebnis ist ein zweidimensionales Bild. Die Tiefeninformationen werden von der Phase des Lichts übertragen. Die Technik der Holographie zeichnet nun auch diese Information mit auf wodurch ein dreidimensionales Bild entsteht.\\

Im Grunde wird ein Hologramm durch einen in zwei Laserwege aufgespaltenen kohärenten Laser in einem lichtempfindlichen Medium gebildet. Dabei benutzt man, dass einer dieser Laserwege an einem Objekt reflektiert wird (der sog. Objektstrahl) und der andere Laserweg (der sog Referenzstrahl) mit diesem ein Interferenzmuster erzeugt.
Wird nun dieses Interferenzmuster auf einer Photoplatte gespeichert und entwickelt, erhält man ein Intensitätsmuster bzw. ein Amplituden- oder Phasenhologramm, das alle nötigen Informationen enthält welche zur Rekonstruktion eines räumlichen Bildes des Objekts nötig sind.

\section{Physikalische Grundlagen der Holographie}

\subsection{Das Michelson-Interferometer}
Wie sich später herausstellen wird ist in der Holographie ein Mechanisch besonders stabiler Aufbau von Nöten. Um eine Vorstellung von der im Experiment gegeben Atabilität des Aufbaus zu bekommen kann ein Michelson-Interferometer genutzt werden. 

Hierbei wird kohärentes Licht auf einen Halbdurchlässigen Spielgel der in einem Winkel von $\alpha = \frac{\pi}{4}$ ausgerichtet ist geleitet und von beiden möglichen Lichtwegen durch weitere Spiegel reflektiert. Anschließend wird das licht der zwei Lichtwege wieder auf dem Halbdurchlässigen Spiegel zusammengeführt und zu einem Schirm/Detektor geleitet.

Durch die verschiedenen Lichtwege kommt es auf dem Schirm/Detektor zu Interferenzerscheinungen der beiden Lichtwellen. Dabei haben selbst kleinste veränderungen der Weglängen, die z.B. durch Erschütterungen entstehen können auswirkungen auf das Interferenzmuster. Somit kann man an Hand des Michelson-Interferometers selbst kleinste veränderungen von Interferenzmustren sichtbar machen, die in der Holographie möglicht vermieden werden sollen.


\newpage
\subsection{Verwendete Arbeitsmaterialien}
\begin{itemize}
\item schwingungsgedämpfter Tisch 
\item He-Ne-Laser, 10 mW, pol., rot: $\lambda = 632.8$ nm, mit Justierhalterung und Netzgerät 
\item Filterhalterung, dazu: Graufilter in verschiedenen Dichten 
\item Wedge-Filter, zirkular (variabler Strahlteiler) 
\item Strahlteiler (1:1) 
\item 2 Achromaten $f = 400$ mm ($D = 50$ mm) 
\item 2 Achromaten $f = 160$ mm ($D = 40$ mm) 
\item 4 Planspiegel 
\item 2 justierbare Raumfilter mit Mikroskopobjektiv (10x; $f = 16.9$ mm) und 15$\mu$m Lochblende oder Mikroskopobjektiv (16x; $f = 10.8$ mm) und 25 $\mu$m Lochblende 
\item 2 Irisblenden D/A 37.0 mm, $D_{max} = 25.0$ mm 
\item 2 Irisblenden D/A 70.0 mm, $D_{max} = 50$ mm 
\item Elektronischer Verschluss mit Verschlusskontrolleinheit 
\item Belichtungsmesser 
\item 1 Fotodiode, Kabel, Stromverstärker, Oszilloskop, Rechner 
\item Filmmaterial: Slavich PFG-01, dazu: Planfilmhalterung, 
\item Schirme 
\item Dunkelkammer mit entsprechender Ausstattung 
Zu allen optischen Komponenten gehören Montagestäbe, Halterungen und Magnethalter. 
\end{itemize}
