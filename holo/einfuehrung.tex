\section{Physikalische Grundlagen}

\subsection{Holographie}
Wenn man ein Photo eines Gegenstands aufnimmt, so wird mittels inkohärentem Licht die Information auf eine Photoplatte bzw. einem Chip gespeichert. Dabei gehen allerdings die Tiefeninformationen des Objekts verloren, weil nur die Amplitude des Lichts gespeichert wird. Das Ergebnis ist ein zweidimensionales Bild. Die Tiefeninformationen werden von der Phase des Lichts übertragen. Die Technik der Holographie zeichnet nun auch diese Information mit auf wodurch ein dreidimensionales Bild entsteht.\\

Im Grunde wird ein Hologramm durch zwei kohärente Laserstrahlen in einem lichtempfindlichen Medium gebildet. Dabei benutzt man, dass einer dieser Laserstrahlen an einem Objekt gestreut wird (der sog. Objektstrahl) und der andere Laserstrahl (der sog Referenzstrahl) mit diesem ein Interferenzmuster erzeugt.
Wird nun dieses Interferenzmuster auf einer Photoplatte gespeichert und entwickelt, erhält man ein Intensitätsmuster bzw. ein Amplituden- oder Phasenhologramm, das alle nötigen Informationen enthält welche zur Rekonstruktion eines räumlichen Bildes des Objekts nötig sind.


\newpage
\subsection{Verwendete Arbeitsmaterialien}
\begin{itemize}
\item schwingungsgedämpfter Tisch 
\item He-Ne-Laser, 10 mW, pol., rot: $\lambda = 632.8$ nm, mit Justierhalterung und Netzgerät 
\item Filterhalterung, dazu: Graufilter in verschiedenen Dichten 
\item Wedge-Filter, zirkular (variabler Strahlteiler) 
\item Strahlteiler (1:1) 
\item 2 Achromaten $f = 400$ mm ($D = 50$ mm) 
\item 2 Achromaten $f = 160$ mm ($D = 40$ mm) 
\item 4 Planspiegel 
\item 2 justierbare Raumfilter mit Mikroskopobjektiv (10x; $f = 16.9$ mm) und 15$\mu$m Lochblende oder Mikroskopobjektiv (16x; $f = 10.8$ mm) und 25 $\mu$m Lochblende 
\item 2 Irisblenden D/A 37.0 mm, $D_{max} = 25.0$ mm 
\item 2 Irisblenden D/A 70.0 mm, $D_{max} = 50$ mm 
\item Elektronischer Verschluss mit Verschlusskontrolleinheit 
\item Belichtungsmesser 
\item 1 Fotodiode, Kabel, Stromverstärker, Oszilloskop, Rechner 
\item Filmmaterial: Slavich PFG-01, dazu: Planfilmhalterung, 
\item Schirme 
\item Dunkelkammer mit entsprechender Ausstattung 
Zu allen optischen Komponenten gehören Montagestäbe, Halterungen und Magnethalter. 
\end{itemize}