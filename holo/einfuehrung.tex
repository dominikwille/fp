\section{Vorwort}
Wenn man ein Photo eines Gegenstands aufnimmt, so wird mittels inkohärentem Licht, welches von dem Objekt reflektiert wird, die Information auf eine Photoplatte bzw. einem Chip gespeichert. Dabei gehen allerdings die Tiefeninformationen des Objekts verloren, weil nur die Amplitude des Lichts gespeichert wird. Das Ergebnis ist ein zweidimensionales Bild. Die Tiefeninformationen werden von der Phase des Lichts übertragen. Die Technik der Holographie zeichnet nun auch diese Information mit auf, wodurch ein dreidimensionales Bild entsteht.\\

Im Grunde wird ein Hologramm durch zwei kohärente Laserstrahlen in einem lichtempfindlichen Medium gebildet. Dabei benutzt man, dass einer dieser Laserstrahlen an einem Objekt gestreut wird (der sog. Objektstrahl) und der andere Laserstrahl (der sog Referenzstrahl) mit diesem ein Interferenzmuster erzeugt. Durch diese Art der Überlagerung von Licht ist es möglich, auch die Phaseninformation zu speichern und somit ein dreidimensionales Abbild des Objekts zu erzeugen.\\

Zwei Probleme der Holographie sind allerdings die Kohärenz des Lichts und die Empfindlichkeit der Apparatur. Weißes Licht besitzt in der Regel eine Kohärenzlänge von $1\mu m$. Da die Apparatur jedoch bei weitem Größer ist, braucht man Licht dessen Kohärenz auch entsprechend Groß ist. Um dieses Problem zu lösen, kam es zum Einsatz von Lasern. Somit ist die Entwicklung der Holographie direkt von der der Laser abhängig.
Das zweite Problem der Empfindlichkeit wird dadurch gelöst, dass holographische Effekte nur in Laboratorien beobachtet bzw. hergestellt werden können. An einem alltäglichen Gebrauch ist mit dem heutigen Stand der Technik noch nicht zu denken.

\section{Physikalische Grundlagen}

\subsection{Hologrammarten}
Im Laufe der Zeit haben sich verschiedene Arten von Hologrammen entwickelt. Dabei unterscheidet man zum einen Hologramme je nach Lage des Objekts zwischen Transmissions- und Reflexionshologrammen. Zum anderen kommt es drauf an, wie die Informationen auf der Photoplatte gespeichert wird.\\

Die einfachste Art ist die Sichtlinien- bzw. Transmissionsholographie. Dabei wird ein Laserstrahl aufgeweitet und durch das teildurchsichtige Objekt auf eine Photoplatte geworfen. Das Licht, welches direkt durch das Objekt geht, legt durch Beugung und Streuung an der Objektoberfläche einen optisch längeren Weg zurück als das Licht, welches an dem Objekt vorbei geht. Der so entstehende Phasenunterschied verursacht eine Interferenz zwischen den beiden Strahlen, welche von der gleichen Seite auf die Photoplatte treffen. Das so entstandene Hologramm wird auch als dünnes Hologramm bezeichnet.\\

Wenn man nun kein durchsichtiges Objekt abbilden möchte, wird dies natürlich schwer. Deswegen entwickelte man eine Methode, bei der die Objektwelle durch Reflexion am Objekt die Photoplatte erreicht. Nun gibt es da allerdings zwei Methoden ein sogenanntes Reflexionshologramm herzustellen.
Die eine Möglichkeit ist, den Laserstrahl mithilfe eines Strahlteilers in eine Objektwelle und eine Referenzwelle zu teilen. Die Objektwelle wird an dem Objekt reflektiert, wodurch der Phasenunterschied erzeugt wird. Der Referenzstrahl gelangt ungehindert auf die Photoplatte, wo er mit dem Objektstrahl interferieren kann. Da auch hier wieder beide Strahlen von der gleichen Seite auf die Photoplatte trifft, ist das entstandene Hologramm wieder ein dünnes Hologramm.
Bei der anderen Methode entsteht ein sogenanntes Weißlichtreflexionshologramm. Dabei trifft ein Teil des Lasers durch die anfangs durchsichtige Photoplatte als Referenzstrahl auf das Objekt dahinter und wird an diesem reflektiert. Der reflektierte Strahl trifft nun von der anderen Seite auf die Photoplatte und interferiert nun als Objektstrahl mit der Referenzwelle. Dabei entstehen in der Photoplatte stehende Wellen wodurch in der gesamten Platte Informationen gespeichert werden. Das daraus entstandene Hologramm wird als dickes Hologramm bezeichnet.

\subsection{Das Michelson-Interferometer}
Wie sich später herausstellen wird ist in der Holographie ein Mechanisch besonders stabiler Aufbau von Nöten. Um eine Vorstellung von der im Experiment gegeben Atabilität des Aufbaus zu bekommen kann ein Michelson-Interferometer genutzt werden. 

Hierbei wird kohärentes Licht auf einen Halbdurchlässigen Spielgel der in einem Winkel von $\alpha = \frac{\pi}{4}$ ausgerichtet ist geleitet und von beiden möglichen Lichtwegen durch weitere Spiegel reflektiert. Anschließend wird das licht der zwei Lichtwege wieder auf dem Halbdurchlässigen Spiegel zusammengeführt und zu einem Schirm/Detektor geleitet.

\myImage[8cm]{img/michelson}{Schematische Darstellung des Michelson-Interferometers}

Durch die verschiedenen Lichtwege kommt es auf dem Schirm/Detektor zu Interferenzerscheinungen der beiden Lichtwellen. Dabei haben selbst kleinste veränderungen der Weglängen, die z.B. durch Erschütterungen entstehen können auswirkungen auf das Interferenzmuster. Somit kann man an Hand des Michelson-Interferometers selbst kleinste veränderungen von Interferenzmustren sichtbar machen, die in der Holographie möglicht vermieden werden sollen.

Zu beachten ist dabei aber sicherlich, dass der Wegunterschied kleiner als die Kohärenzlänge $L$ sein muss. Wird die Kohärenzlänge überschritten, so ist keine Interferenz mehr sichtbar.
\subsubsection{Bestimmung der Kohärenzlänge}
Aus der gemessenen Kontrastfunktion,
\begin{align}
K(z) = \frac{I_{max}-I_{min}}{I_{max}+I_{min}} = \frac{2\sqrt{I_1 \cdot I_2}}{I_1+I_2} \cdot Y
\end{align}
bzw. deren Nullstellen kann man Rückschlüsse auf die Kohärenzlänge machen. Es gilt:
\begin{align}
L = \text{Maximale Weglänge mit Interferenz}
\end{align}

\subsection{Holographie}
In der Holographie nutzt man eine Photoplatte um ein Interferenzmuster aus der \textit{Objektwelle} $S$ und der \textit{Referenzwelle} $R$ zu speichern, dabei werden nicht nur die Intensitäten sonden auch die Phasen oder Amplituden gespeichert.

Die Intensität in/auf der Photoplatte ergibt sich aus:
\begin{align}
E(\vec{r}, t) &= S 
\end{align}

\newpage
\subsection{Verwendete Arbeitsmaterialien}
\begin{itemize}
\item schwingungsgedämpfter Tisch 
\item He-Ne-Laser, 10 mW, pol., rot: $\lambda = 632.8$ nm, mit Justierhalterung und Netzgerät 
\item Filterhalterung, dazu: Graufilter in verschiedenen Dichten 
\item Wedge-Filter, zirkular (variabler Strahlteiler) 
\item Strahlteiler (1:1) 
\item 2 Achromaten $f = 400$ mm ($D = 50$ mm) 
\item 2 Achromaten $f = 160$ mm ($D = 40$ mm) 
\item 4 Planspiegel 
\item 2 justierbare Raumfilter mit Mikroskopobjektiv (10x; $f = 16.9$ mm) und 15$\mu$m Lochblende oder Mikroskopobjektiv (16x; $f = 10.8$ mm) und 25 $\mu$m Lochblende 
\item 2 Irisblenden D/A 37.0 mm, $D_{max} = 25.0$ mm 
\item 2 Irisblenden D/A 70.0 mm, $D_{max} = 50$ mm 
\item Elektronischer Verschluss mit Verschlusskontrolleinheit 
\item Belichtungsmesser 
\item 1 Fotodiode, Kabel, Stromverstärker, Oszilloskop, Rechner 
\item Filmmaterial: Slavich PFG-01, dazu: Planfilmhalterung, 
\item Schirme 
\item Dunkelkammer mit entsprechender Ausstattung 
Zu allen optischen Komponenten gehören Montagestäbe, Halterungen und Magnethalter. 
\end{itemize}
