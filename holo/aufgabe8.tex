\subsection*{Aufgabe 8 - Stellung des Objekthalters}
Dass der Objekthalter schräg auf der Konstruktion steht ist an sich relativ einfach zu erklären. Wir wollen mit einem Hologramm ein Bild anfertigen, in dem auch die Tiefeninformation gespeichert ist. Um die Tiefeninformation zu erhalten, muss das Objekt so stehen, dass wir nicht nur die Frontseite sehen. Also stellt man die Ablagefläche des Objekts einfach schräg und erhält so neben der Frontsicht auch noch eine gewisse Draufsicht. Um möglichst hochwertige Hologramme zu erhalten, sollten die Objekte aus einem  möglichst hellen Material sein. Da weiße Oberflächen Licht sehr stark in alle Richtungen reflektiert, eignen sich Papiermodelle o.ä. am besten zum erstellen eines Hologramms.