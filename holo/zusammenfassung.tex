\section{Zusammenfassung und Diskussion}

\subsection{Aufgabe 1}
Zu dieser Aufgabe lässt sich schwer etwas zusammenfassend schreiben, da wir nur die vorhandenen Hologramme betrachten sollten. Dies ist uns soweit gelungen.

\subsection{Aufgabe 2}
Da wir schon aus früheren Praktika mittlerweile gut im Umgang mit Laserexperimenten und deren Aufbau und Kalibrierung geschult sind, hatten wir soweit keine Probleme ein Michelson-Interferometer aufzubauen. Wir waren also mit dieser Aufgabe ziemlich schnell fertig und hatten eine recht gute Kalibrierung.

\subsection{Aufgabe 3}
Da wir uns während der Aufnahmen der Hologramme ruhig verhalten hatten und auch immer abwarteten  bis um das Labor herum alles ruhig ist, gelang es uns gute Hologramme aufzunehmen.

\subsection{Aufgabe 4}
Bei der Erstellung der Hologramme traten keinerlei Probleme oder Komplikationen auf. Wir bereiteten die Modelle in zügiger Zusammenarbeit vor und schnitten die Fotoplatten zurecht. Das auswechseln der Fotoplatten verlief auch reibungslos und so waren wir ziemlich schnell mit dieser Arbeit fertig.

\subsection{Aufgabe 5}
Während der Entwicklung unserer Hologramme traten anfangs ein paar Verständnisprobleme auf, da wir mit dem Ablauf der Nutzung der Lösungen und deren Dauer ein wenig durcheinander kamen. Aber sonst verlief auch an dieser Stelle der Versuch zufriedenstellend.