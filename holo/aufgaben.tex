\section{Aufgaben}

\subsection*{Aufgabe 1}
Rekonstruieren Sie mit einem der vorhandenen Hologramme die dort gespeicherte Gegenstandsinformation (eventuell auf dem Hologramm vorhandene Beschriftung zum Beobachter). Wenn Sie das virtuelle Bild gefunden haben, wo müsste das reelle Bild liegen? Beschreiben und begründen Sie Ihre Beobachtungen

\subsection*{Aufgabe 2}
Bauen Sie mit den zur Verfügung stehenden Komponenten ein Michelson-Interferometer auf. Überprüfen Sie die mechanische Stabilität Ihres Aufbaus in Bezug auf kurzzeitige Störungen. 

\subsection*{Aufgabe 3}
Was können Sie hieraus in Bezug auf die von Ihnen zu erstellenden Hologramme schließen? 

\subsection*{Aufgabe 4}
Diskutieren Sie den Aufbau und die geplante Belichtung für die Herstellung von Weißlichtreflexionshologrammen mit Ihrem Betreuer. Montieren Sie den Holografiefilm und nehmen Sie mit zweien Ihrer Objekte Hologrammeauf. (Bitte machen Sie von jedem Objekt zwei identische Expositionen! Denken Sie daran, die Filme vor der 
Belichtung geeignet zu markieren, insbesondere falls Sie verschiedene Belichtungszeiten wählen.) 

\subsection*{Aufgabe 5}
Entwickeln Sie Ihre Hologramme in der Dunkelkammer. Nach einer Trocknungszeit von  ca. 60 min überprüfen Sie das Ergebnis und diskutieren Sie es mit Ihrem Betreuer. Während des Trocknens schrumpft die Gelatineschicht des Filmmaterials. Welche Auswirkungen hat das auf das fertige Hologramm?

\subsection*{Aufgabe 6}
Geben Sie eine mathematische Behandlung für das Speichern und Auslesen von Hologrammen unter Verwendung der Wellenvektoren $k_{referenz}$ 
und $k_{objekt}$. Zeigen Sie explizit die $k_{referenz}$ und $k_{objekt}$ Abhängigkeit von den verschiedenen Termen. 

\subsection*{Aufgabe 7}
Stellen Sie die typischen Prinzipskizzen der Holographie vor.

\subsection*{Aufgabe 8} 
Warum steht der Objekthalter schräg im experimentellen Aufbau für die Aufnahme der Hologramme? 

\subsection*{Aufgabe 9}
Die Abhängigkeit der Gitterkonstanten von der Farbe des Hologramms ist verknüpft mit einer Schrumpfung der Filme. Nehmen Sie an, dass die Braggbedingung für eine gewisse Gitterkonstante d beim Winkel $r = 30^{\circ}$ die passende Wellenlänge 633 nm hat. Berechnen Sie dann welche Wellenlänge (Farbe) man von diesen Hologrammen nach der Gitterschrumpfung auf $d = 560$ nm sieht? 

\subsection*{Aufgabe 10}
Messen Sie den Selbstkohärenzgrad und schätzen die Kohärenzlänge für das verwendete He-Ne-Laserlicht ab. 

\subsection*{Aufgabe 11}
Schätzen Sie im Michelson-Interferometer die Verschiebung des Interferenzbildes (in Vergleich mit $\lambda/2$ ) durch verschieden mechanische Störungen, auch von der Stimme ab. Wie lang ist die Beruhigungszeit des Aufbaus?