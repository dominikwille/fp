\subsection*{Aufgabe 1 - Rekonstruktion der Hologramme}
Um die vorhandenen Hologramme zu untersuchen, legten wir ein Stück schwarzes Papier unter diese und hielten sie unter die Tischlampe. Nach ein bisschen wackeln sahen wir schließlich die gespeicherten Bildinformationen. Wir kamen zu dem Schuss, dass das reelle Bild direkt hinter dem Hologramm liegen muss, denn die Front des Motivs war immer mittig im Bild zu sehen. Kippte man das Hologramm nach Links oder Rechts, so sah man die entsprechende Seite des Motivs. Das Hologramm schimmerte im breiten Farbspektrum und unterschied sich in diesem Punkt stark von dem Original. Als wir später unsere Hologramme herstellten, benutzten wir Gegenstände aus Paper, also mit einer weißen Oberfläche. Auch unsere Hologramme schimmerten später.\\
Somit ist neben der Dimensionalität des Bildes auch eine Differenz in der Farbe festzustellen, was ein Hologramm von einem normalen Foto unterscheidet.