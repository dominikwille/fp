\subsection*{Aufgabe 10 - Selbstkohärenzlänge des He-Ne-Lasers}
Die Kohärenzlänge wird bei diesem Versuch durch die Bestimmung der Kontrastfunktion $K(z)$ bestimmt. Diese wird durch Messung der relativen Intensitäten $I_{rel}$ des Lasers bei verschiedenen Weglängendifferenzen $\Delta x$ bestimmt.

\begin{center}
\begin{tabular}{c|c|c}
$\Delta x\, \text{[cm]}$ & $I_{rel}\,\text{[\%]}$ & $\Delta I_{rel}\, \text{[\%]}$ \\\hline
\(4\) & \(82.0\) & \(0.071\) \\ 
\(12\) & \(59.0\) & \(0.0595\) \\ 
\(20\) & \(38.0\) & \(0.049\) \\ 
\(28\) & \(14.0\) & \(0.037\) \\ 
\(36\) & \(05.8\) & \(0.0329\) \\ 
\(44\) & \(11.3\) & \(0.03565\) \\ 
\(52\) & \(18.0\) & \(0.039\) \\ 
\(60\) & \(13.6\) & \(0.0368\) \\ 
\(68\) & \(34.2\) & \(0.0471\) \\ 
\(76\) & \(58.8\) & \(0.0594\) \\ 
\(84\) & \(81.3\) & \(0.07065\)
\end{tabular}
\captionof{table}{Bestimmung der Kontrastfunktion}
\end{center}
\myImage[16cm]{img/a10}{Bestimmung der Kontrastfunktion}

Die Kohärenzlänge $L$ kann nun durch die betimmung der Minima bestimmt werden zu

\begin{align}
L = (40 \pm 5)\, \text{cm}
\end{align}
