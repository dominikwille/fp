\subsection*{Aufgabe 10 - Selbstkohärenzlänge des He-Ne-Lasers}
Für diesen Versuch nahmen wir mithilfe des Computers eine Messreihe auf, wobei wir den Abstand des verschiebbaren Spiegels des Michelson-Interferometers in 2 cm Schritten veränderten. Dadurch betrug die Änderung der Weglänge jeweils 4 cm. Am Anfang war der bewegliche Spiegel 40 cm von dem Strahteiler entfernt, rückte jedoch mit jedem Schritt näher. Wir nahmen die Messwerte mit dem zur Verfügung stehenden Programm auf, nachdem wir auf dem Schirm vor dem Photosensor ein Interferenzmuster zustande bekamen. 
Dabei verlief die Messung immer gleich. Als erstes starteten wir die Messung mittels des Programm, ließen den Laser allerdings ausgeschaltet um eine Referenzlinie zu bekommen. Nach zwei bis drei Sekunden schalteten wir den Laser ein und wackelten am Gehäuse des Aufbaus. Das Programm registrierte dadurch den Wechsel zwischen Maxima und Minima des Interferenzmusters auf dem Photosensor. Als Ergebnis bekamen wir ein Bild welches in Abb. 8 in dem Abschnitt "1. Gesamtbild" zu erkennen ist. Dieses Bild wurde anschließend in ein zweites Programm geladen (welches in Abb. 8 dargestellt wird) und Bearbeitet. 

\myImage[16cm]{img/Nulllinie}{Ausschnitt des Programms zur Bearbeitung der Messdaten}

Als erstes musste man mithilfe der roten und blauen Grenze vertikaler Anordnung aus dem Ausschnitt $"$1. Gesamtbild$"$ die Höhe der Nulllinie bestimmen. Das Ergebnis wird in dem kleinen Fenster auf der rechten Seite mit der Beschriftung $"$Nulllinie$"$ angezeigt. Der Wert variierte während der gesamten Messung nur um einen sehr geringen Teil.\\
Danach verschob man die beiden Grenzlinien in den Ausschnitt, in welchem die $"$Störung$"$ möglichst konstant war (wie gerade in dem Abschnitt $"$1. Gesamtbild$"$ zu sehen ist). Durch klicken der Schaltfläche $"$Ausgewählten Bereich analysieren$"$ wurde der Bereich in das Fenster mit der Überschrift $"$Ausschnitt zwischen den Cursorn aus (1)$"$ angezeigt. Danach klickte man auf die Schaltfläche $"$Weiter mit der Analyse der Peaks$"$, wodurch der Ausschnitt aus Fenster 2 in das Fenster mit der Überschrift $"$3. Analyse der Peaks aus (2)$"$ eingefügt wird. Nun musste man wieder einen roten und einen blauen Cursor, diesmal allerdings horizontal angeordnet, zur Mitte des Bildes hin ziehen. Dadurch wurden die Spitzen der Peaks mit roten bzw. blauen Punkten markiert. Wenn wir das Gefühl hatten, genügend Spitzen markiert zu haben, entnahmen wir dem Feld $"$Sichtbarkeit$"$ einen Wert, welcher die Intensität des Interferenzmusters in Prozent angibt.

Je nach Abstand des Spiegels variierte dieser Wert. Unsere Messwerte werden in der folgenden Tabelle aufgeführt.

Die Schwierigkeit bestand darin, dass bei bestimmten Abständen es schwer wurde ein Interferenzmuster auszumachen.
Die Kohärenzlänge wird bei diesem Versuch durch die Bestimmung der Kontrastfunktion $K(z)$ bestimmt. Diese wird durch Messung der relativen Intensitäten des Lasers bei verschiedenen Weglängendifferenzen $\Delta x$ bestimmt.

\myImage[15cm]{img/a10}{Bestimmung der Kontrastfunktion}

Die Kohärenzlänge $L$ kann nun durch die Bestimmung der Minima bestimmt werden zu

\begin{align}
L = (40 \pm 5)\, \text{cm}
\end{align}
