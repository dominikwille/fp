\subsection*{Aufgabe 7 - Prinzipskizzen der Holographie}
Zur Lösung der Aufgabe verweise ich auf die Einführung (Seite 3, 2.1.1 und 2.1.2). \\
<<<<<<< HEAD
Bei Transmissionshologramme transmitiert das einfallende Licht, so dass sich Betrachter und Lichtquelle auf unterschiedlichen Seiten des Films befinden müssen. Bei der Aufnahme treffen Referenzstrahl und Objektstrahl von der gleichen Seite auf den Film und erzeugen dort ein Interferenzmuster. Für die Rekonstruktion des Bildes muss das Transmissionshologramm mit einer kohärenten Lichtquelle durchleuchtet werden.
=======
Bei Transmissionshologramme transmittieren das einfallende Licht, so dass sich Betrachter und Lichtquelle auf unterschiedlichen Seiten des Films befinden müssen. Bei der Aufnahme treffen Referenzstrahl und Objektstrahl von der gleichen Seite auf den Film und erzeugen dort ein Interferenzmuster. Für die Rekonstruktion des Bildes muss das Transmissionshologramm mit einer kohärenten Lichtquelle durchleuchtet werden.
>>>>>>> b6dacf3618fb14eb187c0083e9b6a7013d749738

\myImage[5cm]{img/TransmittHol}{Schema der Transmissionsholographie}

Bei einem Reflexionshologramme reflektiert das Objekt das einfallende Licht, so dass die Lichtquelle im Gegensatz zur Herstellung von Transmissionshologrammen auf der Seite des Betrachters sein kann. Bei Reflexionshologrammen treffen Referenz- und Objektstrahl von unterschiedlichen Seiten auf den holografischen Film. Reflexionshologramme sind immer Volumenhologramme. Im Film entstehen verschiedene Netzebenen, die durch die von den Interferenzmaxima belichteten Stellen des Aufnahmematerials gehen. Bei der Rekonstruktion reflektieren die Netzebenen das einfallende Licht so zurück, so dass ein Bild des Gegenstands entsteht.\\
Nur wenn die Bragggleichung $n \cdot \lambda = 2 \cdot d \cdot sin \left( \alpha \right)$ erfüllt ist, kann das im Winkel $\alpha$ einfallende Licht mit der Wellenlänge $\lambda$ am Film mit dem Netzebenenabstand d reflektiert werden. Bei Weißlichtreflexionshologrammen hängt deshalb die Farbe des Hologramms vom Einfallswinkel des Lichts ab.

\myImage[5cm]{img/ReflexHol}{Schema der Reflexionsholographie}

Ein vereinfachtes Verfahren zur Herstellung von Reflexionshologrammen ist die Denisjuk-Holografie. Im Gegensatz zur normalen Reflexionsholografie wird der Laserstrahl hier nicht geteilt, sondern durchleuchtet als Referenzstrahl den holografischen Film.
Hinter dem Film befindet sich das aufzunehmende Objekt, das den Referenzstrahl teilweise zurück zum Film reflektiert.\\
Der entstehende Objektstrahl und der Referenzstrahl treffen von unterschiedlichen Seiten auf den Film und erzeugen eine Interferenzmuster. Denisjuk-Hologramme können wie alle Reflexionshologramme unter weißem Licht rekonstruiert werden.

\myImage[5cm]{img/DenisjukHol}{Schema der Weißlichtholographie}