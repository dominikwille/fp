\section{Zusammenfassung und Diskussion}
In diesem Versuch sollten wir die Rayligh-Streuung in Luft und Vakuum mithilfe des CRDS untersuchen. Wir hatten einige Probleme bei der Justage, der Versuch hingegen verlief ohne Probleme.\\
Die erhaltenen Werte für die evakuierte Kavität sind etwas kleiner als erwartet. Das gleiche trifft auf die Ergebnisse der Messung mit Luft in der Kavität zu. Es liegt nahe, dass es sich um einen systematischen Fehler handelt, der dadurch hervorgerufen wurde, dass der Aufbau nicht richtig justiert wurde.
Des weiteren stellten wir fest, dass der berechnete Streukoeffizient von $\beta =   (4,31 \pm 0,04) \cdot 10^{-5} m^{-1}$ ebenfalls von dem erwarteten Ergebnis abweicht. Das bestärkte uns in der Annahme, dass es sich um einen systematischen Fehler handeln muss.\\
Alles in allem lässt sich also sagen, dass die Messung selbst erfolgreich verlaufen ist, die Justage hingegen war nicht erfolgreich.