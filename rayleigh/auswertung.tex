\section{Auswertung}

Die Auswertung wurde mit Python gemacht. Die Werte wurden nochmal genauer gefittet und in einem Graphen dargestellt. Als Fitfunktion wurde eine exponentielle Funktion mit den gesuchten Parametern verwendet. 

\subsection{Evakuierte Kavität}

\myImage[9cm]{img/mess1vak}{evakuierte Messung}

Im Bild sind die Originalwerte, sowie die Fitfunktion zu erkennen. Daraus ergab sich der folgende Wert für die Abklingkonstante:
\begin{equation}
\notag
\tau_{0}= (10.44 \pm 0.03)\mu s
\end{equation}

Der Fit zeigt einen genaueren Verlauf der Funktion mit der Abklingkonstante. Während des Experimentes mit dem Labview ergab sich der Wert dafür 
\begin{equation}
\notag
\tau_{0/class}= (10.25 \pm 0.03)\mu s
\end{equation}

\subsection{Luftgefüllte Kavität}

\myImage[9cm]{img/mess1luft}{Luftgefüllte Messung}

Für die luftgefüllte Messung bekamen wir nach dem Fitten der Parameter an eine exponentielle Funktion den folgenden Wert:

\begin{equation}
\notag
\tau_{luft}= (9.05 \pm 0.04)\mu s
\end{equation}

Der im Experimentierwert während des Experiments ergab sich mit Labview zu:
\begin{equation}
\notag
\tau_{luft/class}= (9.06 \pm 0.04)\mu s
\end{equation}

Nun wird noch überprüft, in wie weit die Werte mit dem Streukoeffizienten verträglich sind.

\subsection{Bestimmung des Streukoeffizienten}

Der theoretische Wert für den Streukoeffizient wird nach (6) berechnet. Der Literaturwert für den Brechungsindex n beträgt 1,000276. Für die Anzahl der Teilchen in der Luft sind wir von dem thermischen Gleichgewicht ausgegangen, d.h. $N = \frac{p}{k_B T}$. Der Druck betrug $(1018.1 \pm 0.5)hpa$. Die Temperatur war die normale Raumtemperatur von $(22 \pm 1)^{\circ}C$ Die Wellenlänge von dem Laser war durch $(405 \pm 2)nm$ vorgegeben. Also lautet der theoretische Wert für den Streukoeffizient

\begin{equation}
\notag
\beta(\lambda)= (3.75 \pm 0.04) \times 10^{-5} m^{-1}
\end{equation}

Jetzt kann man den experimentellen Wert mit Gl.(9) bestimmen
\begin{equation}
\notag
\beta(\lambda)= (4.31 \pm 0.04) \times 10^{-5} m^{-1}
\end{equation} 




