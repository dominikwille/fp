\section{Auswertung}

Die Auswertung wurde mit Python gemacht. Die Werte wurden nochmal genauer gefittet und in einem Graphen dargestellt. Als Fitfunktion wurde eine exponentielle Funktion mit den gesuchten Parametern verwendet. 

\subsection{Evakuierte Kavität}

\myImage[9cm]{img/mess1vak}{evakuierte Messung}

Im Bild sind die Originalwerte, sowie die Fitfunktion zu erkennen. Daraus ergab sich der folgende Wert für die Abklingkonstante:
\begin{equation}
\notag
\tau_{0}= (10.44 \pm 0.03)\mu s
\end{equation}

Der Fit zeigt einen genaueren Verlauf der Funktion mit der Abklingkonstante. Während des Experimentes ergab sich der Wert dafür 
\begin{equation}
\notag
\tau_{0/class}= (10.25 \pm 0.03)\mu s
\end{equation}

\subsection{Luftgefüllte Kavität}

\myImage[9cm]{img/mess1luft}{Luftgefüllte Messung}
