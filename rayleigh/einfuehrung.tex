\section{Ziele des Versuchs}
In diesem Versuch wird die Rayleigh-Streuung untersucht, dabei werden Photonen an Teilchen gestreut, die kleiner als die Wellenlänge einfallenden Lichts sind. Zudem werden die Streukoeffizienten durch die Methode der optischen Cavity-Ring-Down-Spektroskopie (CRDS) bestimmt. Im Alltag begegnet man den Effekten der Rayleigh-Streuung bei der Färbung des Himmels. Sie gilt als Teil Theorie der Mie-Streuung.

\section{Physikalische Grundlagen}
\subsection{Mie-Streuung}
Die Mie-Streuung, oder Mie-Theorie, resultiert aus der Lösung der Maxwell-Gleichungen und wurde 1908 von Gustav Mie entwickelt. Sie beschreibt die Streuung von elektromagnetischen Wellen an sphärischen Objekten. John William Strutt, dritter Baron von Rayleigh, entwickelte diese Theorie weiter und beschrieb die Streuung von EM-Wellen an Teilchen, welche kleiner als die Wellenlänge des einfallenden Lichts ist. Unabhängig davon, ob das Licht an feinen Wassertropfen oder an Staubpartikeln streut, gilt für den dimensionslosen Größenparameter $x$ in Abhängigkeit der Wellenlänge $\lambda$ und dem Radius der Teilchen $r$:

\begin{equation}
x = \frac{2\pi r}{\lambda}
\end{equation}

Für den Fall, dass x$\approx$1 gilt, handelt es sich um Mie-Streuung. Dabei schwingen die Elektronen nicht in Phase und der Zusammenhang zwischen Lichtstreuung und dem Größenparameter $x$ ist aufgrund von Resonanzen o.ä. komplizierter. Bei x<<1 handelt es sich um Rayleigh-Streuung. In diesem Fall schwingen die Elektronen in Phase und das Teilchen kann als Herz'scher Dipol angenähert werden. 

\subsection{Rayleigh-Streuung}

\subsection{Wirkungsquerschnitt}

\subsection{Cavity-Ring-Down-Spektroskopie}