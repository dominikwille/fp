\section{Physikalische Grundlagen}

Nach der klassischen Theorie der Dispersion, schwingen im Feld der eingestrahlten Lichtwelle die positiven und negativen Bausteine der Moleküle gegeneinander, sodass jedes einzelne Molekül einem Hertz'schen Dipol entspricht. Die Dipole strahlen ihrerseits Licht mit der bekannten Strahlungscharakteristik ab; es tritt also eine Lichtstreuung auf. Die abgestrahlte Feldstärke ist proportional dem Quadrat der Schwingungsfrequenz, die gleich der Frequenz des einfallenden Lichtes ist. Die abgestrahlte Leistung, die sich aus dem Quadrat der Feldstärke ergibt, ist damit proportional der vierten Potenz der Frequenz der reziproken Wellenlänge.
Da der Durchmesser der Streukörper klein gegen die Lichtwellenlänge sein soll, können die Strahlungsanteile der einzelnen Moleküle ohne Phasenverschiebung addiert werden, sodass die Gesamtstrahlung eines Streukörpers ebenfalls der Strahlung eines Hertz'schen Dipols entspricht. Die Intensität des gestreuten Lichtes ist damit umgekehrt proportional der 4. Potenz der Wellenlänge. Da die blauvioletten Wellen (0.451$\mu$m) etwa 0.7-mal so lang sind wie die roten (0.65 $\mu$m), wird das blaue Licht etwa $(1/0.7)^4 =4$ - mal stärker gestreut als das rote. Daher die Blaufärbung.
Das Himmelblau beruht auf dem gleichen Mechanismus wie die Tyndallstreuung.
Streukörper sind hier die Luftmoleküle selbst. Die gestreute Intensität soll hier auf Grund des vorstehend skizzierten Gedankenganges explizit berechnet werden.
Das durch das einfallende Sonnenlicht (Amplitude $E_0$. Kreisfrequenz $\omega_0$) maximale induzierte Dipolmoment p eines Moleküls ergibt sich zu
\begin{align}
P= \frac{P}{N} = \frac{\epsilon_r - 1}{N} \cdot \epsilon_0 \cdot 	E_0
\end{align}
Dabei ist P die Polarisation. d. h. das Gesamtdipolmoment geteilt durch das Volumen, N die Anzahldichte der Moleküle, $\epsilon_r$ die Permittivitätszahl des streuenden Gases. Für die abgestrahlte Feldstärke E gilt in einem Abstand r bei einer Beobachtungsrichtung, die gegen die Polarisationsrichtung der einfallenden Welle um den Winkel $\theta$ geneigt ist (unter Berücksichtigung, dass das Dipolmoment gleich der Dipollänge multipliziert mit der Ladungsamplitude ist):
\begin{align}
E= \frac{\omega^2 p}{4 \pi \epsilon_0 c^2 r} \cdot sin \theta.
\end{align}
Da die Intensitäten den Feldstärkequadraten proportional sind, ergibt sich für die von einem Volumen V, das N V streuende Moleküle enthält, gestreute Intensität J mit $n=\sqrt{\epsilon_r}$
\begin{align}
\frac{J}{J_0}= \frac{NVE^2}{E_0^2} =  \frac{\pi^2V(n^2-1)^2 sin^2\theta}{Nr^2\lambda^4}
\end{align}

Da die Gleichung auf der linken Seite ein maßsystemunabhängiges Verhältnis und auf der rechten Seite nur geometrische Größen enthält, gilt sie für alle Maßsysteme.
Die Abhängigkeit der gestreuten Intensität von der 4. Potenz der reziproken Wellenlänge ergibt, wie bereits erläutert, die Erklärung der blauen Färbung des Himmels.
Die durch den Faktor $sin^2\theta$ gegebene Winkelabhängigkeit gilt nur für polarisiertes Licht. Wie bei einem Hertz’schen Dipol gilt dann. dass bei Beobachtung in Polarizationsrichtung des eingestrahlten Lichtes keine gestreute Intensität registriert wird. Bei Verwendung nicht polarisierten Lichtes ergibt sich, dass bei Beobachtung, senkrecht zur Einstrahlungsrichtung das gestreuten Licht linear polarisiert ist, und zwar senkrecht zur Einstrahlungs - und Beobachtungsrichtung. Dieses experimentelle Ergebnis wird verständlich, wenn das eingestrahlte Licht in einen senkrecht und einen parallel zur Beobachtungsrichtung polarisierten Anteil zerlegt wird.
Die zu GI. (3) führende Überlagerung der Intensitäten der von den einzelnen Molekülen gestreuten Wellen ist deshalb richtig, weil die Streuwellen in der Beobachtungsrichtung statistische Phasen besitzen. Lediglich in Vorwärtsrichtung haben die Streuwellen gegeneinander und gegenüber der einfallenden Welle eine feste Phasenbeziehung. In Vorwärtsrichtung sind daher die Feldstärken der Streuwellen der einfallenden Welle phasenrichtig zu überlagern. Dadurch ergibt sich eine Phasen- und Amplitudenänderung der durchtretenden Welle gegenüber der einfallenden Welle, die durch Brechzahl und Absorptionskoeffizient beschrieben wird.
Durch die Streuung des Lichtes wird zusätzlich zur Absorption durch die Moleküle eine Schwächung des eingestrahlten Lichtes erzeugt. Daher kommt es, dass auch vollkommen farblose, durchsichtige Gase (z. B. Luft] in großen Schichtdicken Licht erheblich schwächen. Man kann die Schwächung der auffallenden Intensität $J_0$ einer bestimmten Wellenlänge $\lambda$ nach Durchlaufen der Schichtdicke x durch die allgemeine Lambert'sche Gleichung darstellen.
\begin{align}
J_x = J_0 e^{-hx}
\end{align}
wo $J_x$, die übriggebliebene Intensität ist. In seiner ausführlichen Theorie hat Lord Rayleigh folgenden Wert für die Schwähungskonstante $\beta$ angegeben:
\begin{align}
\beta(\lambda) = \frac{8\pi^3}{3N\lambda^4}\cdot(n^2-1)^2
\end{align}
Zur Ableitung dieser Gleichung muss man durch Integration von GI. (3) über alle Streurichtungen die alle Streuleistung bestimmen. Daraus kann man dann $\beta$ als Iängenbezogene relative Intensitätsabnahme berechnen.
Die eintretende Schwächung ist wellenabhängig; es tritt also eine Färbung weissen Lichtes nach Durchlaufen einer grossen Gasstrecke ein: Je weiter das weiße Licht in die Schicht eindringt, desto mehr nimmt es einen rötlichen Farbton an.
Dies ist die Erklärung der Morgen- und Abendröte.
Nach der Theorie der Dispersion ist $(n^2 - I )^2$ proportional dem Quadrat der Teilchenzahl N. Dann ist $\beta$ proportional zu N, und dies ist die quantitative Grundlage dafür, dass man aus der Rayleigh-Streuung die Avogadro-Konstante bestimmen kann.
