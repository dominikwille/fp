\section{Versuchsaufbau und Justage}

\myImage[9cm]{img/aufbau}{Aufbau der Messapparatur}

Das Experiment soll folgendermaßen verlaufen:
Der gepulste Laser dringt in die Kavität ein und das transmittierte Signal läuft über einen Filter auf den Photomultiplier. Der Verstärker bringt dieses Signal zum Oszilloskop, wo auch das Lasersignal sichtbar gemacht wird. Die Signale des Oszilloskops werden am PC ausgewertet. Über die Ventile kann die Kavität vakuumiert oder mit Luft gefüllt werden. Zudem lässt sich der Laser über den PC steuern.

Als erstes haben wir den grünen Laser montiert, um die Justage durchzuführen. Danach haben wir die Pumpe für einige Zeit eingeschaltet, um Vakuum in der Kavität zu erzeugen. Nachdem dies erfolgte, fingen wir mit der Justage selbst an. Wichtig dabei war darauf zu achten, dass die beiden Blenden genau auf einer Höhe waren und in derselben Position standen. Dies war aber vom Tutor voreingestellt. Beim Durchgehen des Laserlichtes durch den ersten Spiegel versuchten wir die Reflexion auf den Ausgangsstrahl zu richten. Mit dem zweiten Spiegel sammelten wir alle entstandene Moden auf einen Punkt, so dass sie gebündelt werden. Dieser Vorgang muss für beide Laser wiederholt werden.
Nachdem die Justage für den grünen Laser erfolgte, nahmen wir ihn ab und setzten den violetten an. Der Justagevorgang wurde wiederholt.
Die beiden Laser waren auf cw-Einstellung (continuous wave) eingestellt.

\section{Durchführung}

Der ganze Experiment verlief mit dem violetten Laser im gepulsten Modus. Die Pulsbreite der Signale wurde über den Delay-Generator geregelt. Die Pulsbreite wurde mithilfe des Oszilloskopbildes an die Signale des Photomultipliers angepasst. Um ein rauschfreieres Signal zu erhalten, wurde es am Oszilloskop gemittelt. Die ermittelten Daten konnten wir gleich mit dem Labview-Programm roh ansehen. 

Im ersten Durchgang führten wir die Messungen mit der evakuierter Kavität durch. Es wurden die Laserpulse gestartet und ein Abklingsignal aufgenommen. Dabei stellten wir fest, dass wir keine Photonen am Detektor und somit keine Spannung am Oszilloskop wahrnahmen. Der Grund lag an der verstellten Einstellung von einem der Spiegel. Deswegen musste die Justage wiederholt werden. Beim Zweiten Versuch lagen wir deutlich besser. Mit dem Labview-Programm wurde die Güte des Abklingverhalten überprüft. 
Als nächstes wurde ein Durchgang mit Luft in der Kavität durchgeführt, wobei auch hier zuerst neu justiert wurde. Die Luft wurde durch den Ventil mit Filter eingelassen. Dabei wurde über mehrere Messungen gemittelt.


